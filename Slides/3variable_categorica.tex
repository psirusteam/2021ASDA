% Options for packages loaded elsewhere
\PassOptionsToPackage{unicode}{hyperref}
\PassOptionsToPackage{hyphens}{url}
%
\documentclass[
  ignorenonframetext,
]{beamer}
\usepackage{pgfpages}
\setbeamertemplate{caption}[numbered]
\setbeamertemplate{caption label separator}{: }
\setbeamercolor{caption name}{fg=normal text.fg}
\beamertemplatenavigationsymbolsempty
% Prevent slide breaks in the middle of a paragraph
\widowpenalties 1 10000
\raggedbottom
\setbeamertemplate{part page}{
  \centering
  \begin{beamercolorbox}[sep=16pt,center]{part title}
    \usebeamerfont{part title}\insertpart\par
  \end{beamercolorbox}
}
\setbeamertemplate{section page}{
  \centering
  \begin{beamercolorbox}[sep=12pt,center]{part title}
    \usebeamerfont{section title}\insertsection\par
  \end{beamercolorbox}
}
\setbeamertemplate{subsection page}{
  \centering
  \begin{beamercolorbox}[sep=8pt,center]{part title}
    \usebeamerfont{subsection title}\insertsubsection\par
  \end{beamercolorbox}
}
\AtBeginPart{
  \frame{\partpage}
}
\AtBeginSection{
  \ifbibliography
  \else
    \frame{\sectionpage}
  \fi
}
\AtBeginSubsection{
  \frame{\subsectionpage}
}
\usepackage{amsmath,amssymb}
\usepackage{lmodern}
\usepackage{iftex}
\ifPDFTeX
  \usepackage[T1]{fontenc}
  \usepackage[utf8]{inputenc}
  \usepackage{textcomp} % provide euro and other symbols
\else % if luatex or xetex
  \usepackage{unicode-math}
  \defaultfontfeatures{Scale=MatchLowercase}
  \defaultfontfeatures[\rmfamily]{Ligatures=TeX,Scale=1}
\fi
\usetheme[]{Berkeley}
\usecolortheme{dove}
% Use upquote if available, for straight quotes in verbatim environments
\IfFileExists{upquote.sty}{\usepackage{upquote}}{}
\IfFileExists{microtype.sty}{% use microtype if available
  \usepackage[]{microtype}
  \UseMicrotypeSet[protrusion]{basicmath} % disable protrusion for tt fonts
}{}
\makeatletter
\@ifundefined{KOMAClassName}{% if non-KOMA class
  \IfFileExists{parskip.sty}{%
    \usepackage{parskip}
  }{% else
    \setlength{\parindent}{0pt}
    \setlength{\parskip}{6pt plus 2pt minus 1pt}}
}{% if KOMA class
  \KOMAoptions{parskip=half}}
\makeatother
\usepackage{xcolor}
\IfFileExists{xurl.sty}{\usepackage{xurl}}{} % add URL line breaks if available
\IfFileExists{bookmark.sty}{\usepackage{bookmark}}{\usepackage{hyperref}}
\hypersetup{
  pdftitle={Análisis de encuestas de hogares con R},
  hidelinks,
  pdfcreator={LaTeX via pandoc}}
\urlstyle{same} % disable monospaced font for URLs
\newif\ifbibliography
\usepackage{color}
\usepackage{fancyvrb}
\newcommand{\VerbBar}{|}
\newcommand{\VERB}{\Verb[commandchars=\\\{\}]}
\DefineVerbatimEnvironment{Highlighting}{Verbatim}{commandchars=\\\{\}}
% Add ',fontsize=\small' for more characters per line
\usepackage{framed}
\definecolor{shadecolor}{RGB}{248,248,248}
\newenvironment{Shaded}{\begin{snugshade}}{\end{snugshade}}
\newcommand{\AlertTok}[1]{\textcolor[rgb]{0.94,0.16,0.16}{#1}}
\newcommand{\AnnotationTok}[1]{\textcolor[rgb]{0.56,0.35,0.01}{\textbf{\textit{#1}}}}
\newcommand{\AttributeTok}[1]{\textcolor[rgb]{0.77,0.63,0.00}{#1}}
\newcommand{\BaseNTok}[1]{\textcolor[rgb]{0.00,0.00,0.81}{#1}}
\newcommand{\BuiltInTok}[1]{#1}
\newcommand{\CharTok}[1]{\textcolor[rgb]{0.31,0.60,0.02}{#1}}
\newcommand{\CommentTok}[1]{\textcolor[rgb]{0.56,0.35,0.01}{\textit{#1}}}
\newcommand{\CommentVarTok}[1]{\textcolor[rgb]{0.56,0.35,0.01}{\textbf{\textit{#1}}}}
\newcommand{\ConstantTok}[1]{\textcolor[rgb]{0.00,0.00,0.00}{#1}}
\newcommand{\ControlFlowTok}[1]{\textcolor[rgb]{0.13,0.29,0.53}{\textbf{#1}}}
\newcommand{\DataTypeTok}[1]{\textcolor[rgb]{0.13,0.29,0.53}{#1}}
\newcommand{\DecValTok}[1]{\textcolor[rgb]{0.00,0.00,0.81}{#1}}
\newcommand{\DocumentationTok}[1]{\textcolor[rgb]{0.56,0.35,0.01}{\textbf{\textit{#1}}}}
\newcommand{\ErrorTok}[1]{\textcolor[rgb]{0.64,0.00,0.00}{\textbf{#1}}}
\newcommand{\ExtensionTok}[1]{#1}
\newcommand{\FloatTok}[1]{\textcolor[rgb]{0.00,0.00,0.81}{#1}}
\newcommand{\FunctionTok}[1]{\textcolor[rgb]{0.00,0.00,0.00}{#1}}
\newcommand{\ImportTok}[1]{#1}
\newcommand{\InformationTok}[1]{\textcolor[rgb]{0.56,0.35,0.01}{\textbf{\textit{#1}}}}
\newcommand{\KeywordTok}[1]{\textcolor[rgb]{0.13,0.29,0.53}{\textbf{#1}}}
\newcommand{\NormalTok}[1]{#1}
\newcommand{\OperatorTok}[1]{\textcolor[rgb]{0.81,0.36,0.00}{\textbf{#1}}}
\newcommand{\OtherTok}[1]{\textcolor[rgb]{0.56,0.35,0.01}{#1}}
\newcommand{\PreprocessorTok}[1]{\textcolor[rgb]{0.56,0.35,0.01}{\textit{#1}}}
\newcommand{\RegionMarkerTok}[1]{#1}
\newcommand{\SpecialCharTok}[1]{\textcolor[rgb]{0.00,0.00,0.00}{#1}}
\newcommand{\SpecialStringTok}[1]{\textcolor[rgb]{0.31,0.60,0.02}{#1}}
\newcommand{\StringTok}[1]{\textcolor[rgb]{0.31,0.60,0.02}{#1}}
\newcommand{\VariableTok}[1]{\textcolor[rgb]{0.00,0.00,0.00}{#1}}
\newcommand{\VerbatimStringTok}[1]{\textcolor[rgb]{0.31,0.60,0.02}{#1}}
\newcommand{\WarningTok}[1]{\textcolor[rgb]{0.56,0.35,0.01}{\textbf{\textit{#1}}}}
\usepackage{longtable,booktabs,array}
\usepackage{calc} % for calculating minipage widths
\usepackage{caption}
% Make caption package work with longtable
\makeatletter
\def\fnum@table{\tablename~\thetable}
\makeatother
\setlength{\emergencystretch}{3em} % prevent overfull lines
\providecommand{\tightlist}{%
  \setlength{\itemsep}{0pt}\setlength{\parskip}{0pt}}
\setcounter{secnumdepth}{-\maxdimen} % remove section numbering
\ifLuaTeX
  \usepackage{selnolig}  % disable illegal ligatures
\fi

\title{Análisis de encuestas de hogares con R}
\subtitle{Módulo 2: Análisis de variables categóricas}
\author{}
\date{\vspace{-2.5em}CEPAL - Unidad de Estadísticas Sociales}

\begin{document}
\frame{\titlepage}

\begin{frame}[allowframebreaks]
  \tableofcontents[hideallsubsections]
\end{frame}
\hypertarget{motivaciuxf3n}{%
\section{Motivación}\label{motivaciuxf3n}}

\begin{frame}{Motivación}
Esta sección esta enfocada en los procedimientos para análisis simples
univariados, bivariados y multivariados seleccionados para respuestas de
encuestas categóricas, centrándose en la adaptación de técnicas de
inferencia estadística a datos de encuestas de muestras complejas
\end{frame}

\hypertarget{lectura-y-procesamientos-de-encuestas-con-r}{%
\section{\texorpdfstring{Lectura y procesamientos de encuestas con
\texttt{R}}{Lectura y procesamientos de encuestas con R}}\label{lectura-y-procesamientos-de-encuestas-con-r}}

\begin{frame}[fragile]{Lectura de la base}
\protect\hypertarget{lectura-de-la-base}{}
Iniciemos con la lectura de la encuesta.

\begin{Shaded}
\begin{Highlighting}[]
\NormalTok{encuesta }\OtherTok{\textless{}{-}} \FunctionTok{readRDS}\NormalTok{(}\StringTok{"../Data/encuesta.rds"}\NormalTok{)}
\end{Highlighting}
\end{Shaded}

El paso siguiente es realizar declaración del objeto tipo diseño.
\end{frame}

\begin{frame}[fragile]{Definir diseño de la muestra con \texttt{srvyr}}
\protect\hypertarget{definir-diseuxf1o-de-la-muestra-con-srvyr}{}
Para el desarrollo de la presentación se define el diseño muestral con
la función \emph{as\_survey\_design}.

\begin{Shaded}
\begin{Highlighting}[]
\FunctionTok{options}\NormalTok{(}\AttributeTok{survey.lonely.psu =} \StringTok{"adjust"}\NormalTok{)}
\FunctionTok{library}\NormalTok{(srvyr)}
\NormalTok{diseno }\OtherTok{\textless{}{-}}\NormalTok{ encuesta }\SpecialCharTok{\%\textgreater{}\%} \CommentTok{\# Base de datos.}
  \FunctionTok{as\_survey\_design}\NormalTok{(}
    \AttributeTok{strata =}\NormalTok{ Stratum,  }\CommentTok{\# Id de los estratos.}
    \AttributeTok{ids =}\NormalTok{ PSU,         }\CommentTok{\# Id para las observaciones.}
    \AttributeTok{weights =}\NormalTok{ wk,      }\CommentTok{\# Factores de expansión.}
    \AttributeTok{nest =} \ConstantTok{TRUE}        \CommentTok{\# Valida el anidado dentro}
                       \CommentTok{\# del estrato}
\NormalTok{  )}
\end{Highlighting}
\end{Shaded}
\end{frame}

\begin{frame}[fragile]{Creación de nuevas variables}
\protect\hypertarget{creaciuxf3n-de-nuevas-variables}{}
Durante los análisis de encuesta surge la necesidad de crear nuevas
variables a partir de las existentes, aquí mostramos la definición de
algunas de ellas.\\
\scriptsize

\begin{Shaded}
\begin{Highlighting}[]
\NormalTok{diseno }\OtherTok{\textless{}{-}}\NormalTok{ diseno }\SpecialCharTok{\%\textgreater{}\%} \FunctionTok{mutate}\NormalTok{(}
  \AttributeTok{pobreza =} \FunctionTok{ifelse}\NormalTok{(Poverty }\SpecialCharTok{!=} \StringTok{"NotPoor"}\NormalTok{, }\DecValTok{1}\NormalTok{, }\DecValTok{0}\NormalTok{),}
  \AttributeTok{desempleo =} \FunctionTok{ifelse}\NormalTok{(Employment }\SpecialCharTok{==} \StringTok{"Unemployed"}\NormalTok{, }\DecValTok{1}\NormalTok{, }\DecValTok{0}\NormalTok{),}
  \AttributeTok{edad\_18 =} \FunctionTok{case\_when}\NormalTok{(Age }\SpecialCharTok{\textless{}} \DecValTok{18} \SpecialCharTok{\textasciitilde{}} \StringTok{"\textless{} 18 anios"}\NormalTok{,}
                      \ConstantTok{TRUE} \SpecialCharTok{\textasciitilde{}} \StringTok{"\textgreater{}= 18 anios"}\NormalTok{)}
\NormalTok{)}
\end{Highlighting}
\end{Shaded}

\normalsize

Se ha introducido la función \texttt{case\_when} la cual es una
extensión del a función \texttt{ifelse} que permite crear múltiples
categorías a partir de una o varias condiciones.
\end{frame}

\begin{frame}[fragile]{Dividiendo la muestra en Sub-grupos}
\protect\hypertarget{dividiendo-la-muestra-en-sub-grupos}{}
En ocasiones se desea realizar estimaciones por sub-grupos de la
población, en este caso se extraer 4 sub-grupos de la encuesta.

\begin{Shaded}
\begin{Highlighting}[]
\NormalTok{sub\_Urbano }\OtherTok{\textless{}{-}}\NormalTok{ diseno }\SpecialCharTok{\%\textgreater{}\%}  \FunctionTok{filter}\NormalTok{(Zone }\SpecialCharTok{==} \StringTok{"Urban"}\NormalTok{)}
\NormalTok{sub\_Rural  }\OtherTok{\textless{}{-}}\NormalTok{ diseno }\SpecialCharTok{\%\textgreater{}\%}  \FunctionTok{filter}\NormalTok{(Zone }\SpecialCharTok{==} \StringTok{"Rural"}\NormalTok{)}
\NormalTok{sub\_Mujer  }\OtherTok{\textless{}{-}}\NormalTok{ diseno }\SpecialCharTok{\%\textgreater{}\%}  \FunctionTok{filter}\NormalTok{(Sex }\SpecialCharTok{==} \StringTok{"Female"}\NormalTok{)}
\NormalTok{sub\_Hombre }\OtherTok{\textless{}{-}}\NormalTok{ diseno }\SpecialCharTok{\%\textgreater{}\%}  \FunctionTok{filter}\NormalTok{(Sex }\SpecialCharTok{==} \StringTok{"Male"}\NormalTok{)}
\end{Highlighting}
\end{Shaded}
\end{frame}

\hypertarget{estimaciuxf3n-del-tamauxf1o.}{%
\section{Estimación del tamaño.}\label{estimaciuxf3n-del-tamauxf1o.}}

\begin{frame}{Estimación del tamaño.}
El primer parámetro estimado serán los tamaños de la población y
subpoblaciones.
\end{frame}

\begin{frame}[fragile]{Estimación de tamaño}
\protect\hypertarget{estimaciuxf3n-de-tamauxf1o}{}
\begin{Shaded}
\begin{Highlighting}[]
\NormalTok{(tamano\_zona }\OtherTok{\textless{}{-}}\NormalTok{ diseno }\SpecialCharTok{\%\textgreater{}\%} \FunctionTok{group\_by}\NormalTok{(Zone) }\SpecialCharTok{\%\textgreater{}\%} 
   \FunctionTok{summarise}\NormalTok{(}
     \AttributeTok{n =} \FunctionTok{unweighted}\NormalTok{(}\FunctionTok{n}\NormalTok{()), }\CommentTok{\# Observaciones en la muestra.}
     \AttributeTok{Nd =} \FunctionTok{survey\_total}\NormalTok{(}\AttributeTok{vartype =} \FunctionTok{c}\NormalTok{(}\StringTok{"se"}\NormalTok{,}\StringTok{"ci"}\NormalTok{))))}
\end{Highlighting}
\end{Shaded}

\begin{longtable}[]{@{}lrrrrr@{}}
\toprule
Zone & n & Nd & Nd\_se & Nd\_low & Nd\_upp \\
\midrule
\endhead
Rural & 1297 & 72102 & 3062 & 66039 & 78165 \\
Urban & 1308 & 78164 & 2847 & 72526 & 83802 \\
\bottomrule
\end{longtable}

En la tabla \emph{n} denota el número de observaciones en la muestra por
Zona y \emph{Nd} denota la estimación del total de observaciones en la
población.
\end{frame}

\begin{frame}[fragile]{Estimación de tamaño}
\protect\hypertarget{estimaciuxf3n-de-tamauxf1o-1}{}
Empleando una sintaxis similar es posible estimar el número de personas
en condición de pobreza extrema, pobreza y no pobres.

\begin{Shaded}
\begin{Highlighting}[]
\NormalTok{(tamano\_pobreza }\OtherTok{\textless{}{-}}\NormalTok{ diseno }\SpecialCharTok{\%\textgreater{}\%} \FunctionTok{group\_by}\NormalTok{(Poverty) }\SpecialCharTok{\%\textgreater{}\%} 
   \FunctionTok{summarise}\NormalTok{(}
       \AttributeTok{Nd =} \FunctionTok{survey\_total}\NormalTok{(}\AttributeTok{vartype =} \FunctionTok{c}\NormalTok{(}\StringTok{"se"}\NormalTok{,}\StringTok{"ci"}\NormalTok{))))}
\end{Highlighting}
\end{Shaded}

\begin{longtable}[]{@{}lrrrr@{}}
\toprule
Poverty & Nd & Nd\_se & Nd\_low & Nd\_upp \\
\midrule
\endhead
NotPoor & 91398 & 4395 & 82696 & 100101 \\
Extreme & 21519 & 4949 & 11719 & 31319 \\
Relative & 37349 & 3695 & 30032 & 44666 \\
\bottomrule
\end{longtable}
\end{frame}

\begin{frame}[fragile]{Estimación de tamaño}
\protect\hypertarget{estimaciuxf3n-de-tamauxf1o-2}{}
En forma similar es posible estimar el número de personas debajo de la
linea de pobreza.

\begin{Shaded}
\begin{Highlighting}[]
\NormalTok{(tamano\_pobreza }\OtherTok{\textless{}{-}}\NormalTok{ diseno }\SpecialCharTok{\%\textgreater{}\%} 
   \FunctionTok{group\_by}\NormalTok{(pobreza) }\SpecialCharTok{\%\textgreater{}\%} 
   \FunctionTok{summarise}\NormalTok{(}
       \AttributeTok{Nd =} \FunctionTok{survey\_total}\NormalTok{(}\AttributeTok{vartype =} \FunctionTok{c}\NormalTok{(}\StringTok{"se"}\NormalTok{,}\StringTok{"ci"}\NormalTok{))))}
\end{Highlighting}
\end{Shaded}

\begin{longtable}[]{@{}rrrrr@{}}
\toprule
pobreza & Nd & Nd\_se & Nd\_low & Nd\_upp \\
\midrule
\endhead
0 & 91398 & 4395 & 82696 & 100101 \\
1 & 58868 & 5731 & 47519 & 70216 \\
\bottomrule
\end{longtable}
\end{frame}

\begin{frame}[fragile]{Estimación de tamaño}
\protect\hypertarget{estimaciuxf3n-de-tamauxf1o-3}{}
Otra variable de interés es conocer el estado de ocupación de la
personas.

\begin{Shaded}
\begin{Highlighting}[]
\NormalTok{(tamano\_ocupacion }\OtherTok{\textless{}{-}}\NormalTok{ diseno }\SpecialCharTok{\%\textgreater{}\%} 
   \FunctionTok{group\_by}\NormalTok{(Employment) }\SpecialCharTok{\%\textgreater{}\%} 
   \FunctionTok{summarise}\NormalTok{(}
       \AttributeTok{Nd =} \FunctionTok{survey\_total}\NormalTok{(}\AttributeTok{vartype =} \FunctionTok{c}\NormalTok{(}\StringTok{"se"}\NormalTok{,}\StringTok{"ci"}\NormalTok{))))}
\end{Highlighting}
\end{Shaded}

\begin{longtable}[]{@{}lrrrr@{}}
\toprule
Employment & Nd & Nd\_se & Nd\_low & Nd\_upp \\
\midrule
\endhead
Unemployed & 4635 & 760.6 & 3129 & 6141 \\
Inactive & 41465 & 2162.8 & 37183 & 45748 \\
Employed & 61877 & 2540.1 & 56847 & 66907 \\
NA & 42289 & 2780.0 & 36784 & 47794 \\
\bottomrule
\end{longtable}
\end{frame}

\begin{frame}[fragile]{Estimación de tamaño}
\protect\hypertarget{estimaciuxf3n-de-tamauxf1o-4}{}
Utilizando la función \texttt{group\_by} es posible obtener resultados
por más de un nivel de agregación.

\begin{Shaded}
\begin{Highlighting}[]
\NormalTok{(tamano\_ocupacion\_pobreza }\OtherTok{\textless{}{-}}\NormalTok{ diseno }\SpecialCharTok{\%\textgreater{}\%} 
   \FunctionTok{group\_by}\NormalTok{(Employment, Poverty) }\SpecialCharTok{\%\textgreater{}\%} 
   \FunctionTok{cascade}\NormalTok{(}
       \AttributeTok{Nd =} \FunctionTok{survey\_total}\NormalTok{(}\AttributeTok{vartype =} \FunctionTok{c}\NormalTok{(}\StringTok{"se"}\NormalTok{,}\StringTok{"ci"}\NormalTok{)), }
       \AttributeTok{.fill =} \StringTok{"Total"}\NormalTok{) }\SpecialCharTok{\%\textgreater{}\%} 
   \FunctionTok{data.frame}\NormalTok{()}
\NormalTok{   )}
\end{Highlighting}
\end{Shaded}
\end{frame}

\begin{frame}{Estimación de tamaño}
\protect\hypertarget{estimaciuxf3n-de-tamauxf1o-5}{}
\scriptsize

\begin{longtable}[]{@{}llrrrr@{}}
\toprule
Employment & Poverty & Nd & Nd\_se & Nd\_low & Nd\_upp \\
\midrule
\endhead
Unemployed & NotPoor & 1768 & 405.4 & 965.7 & 2571 \\
Unemployed & Extreme & 1169 & 348.1 & 479.9 & 1859 \\
Unemployed & Relative & 1697 & 457.8 & 790.7 & 2604 \\
Unemployed & Total & 4635 & 760.6 & 3128.7 & 6141 \\
Inactive & NotPoor & 24346 & 1736.3 & 20908.0 & 27784 \\
Inactive & Extreme & 6422 & 1320.7 & 3806.6 & 9037 \\
Inactive & Relative & 10697 & 1460.3 & 7805.9 & 13589 \\
Inactive & Total & 41465 & 2162.8 & 37182.7 & 45748 \\
Employed & NotPoor & 44600 & 2596.2 & 39459.6 & 49741 \\
Employed & Extreme & 5128 & 1121.6 & 2906.6 & 7349 \\
Employed & Relative & 12149 & 1346.6 & 9482.7 & 14816 \\
Employed & Total & 61877 & 2540.1 & 56847.4 & 66907 \\
Total & Total & 150266 & 4181.4 & 141986.5 & 158546 \\
NA & NotPoor & 20684 & 1256.6 & 18195.4 & 23172 \\
NA & Extreme & 8800 & 2979.9 & 2899.7 & 14701 \\
NA & Relative & 12805 & 1551.0 & 9733.9 & 15876 \\
NA & Total & 42289 & 2780.0 & 36784.3 & 47794 \\
\bottomrule
\end{longtable}
\end{frame}

\hypertarget{estimaciuxf3n-de-la-proporciuxf3n.}{%
\section{Estimación de la
proporción.}\label{estimaciuxf3n-de-la-proporciuxf3n.}}

\begin{frame}{Estimación de la proporción.}
Otro resultado de interés es la estimación de las proporciones, dado que
estas entregan una mayor información sobre las tendencias en las
población, siendo de mucha importancia en la toma de decisiones.
\end{frame}

\begin{frame}[fragile]{Estimación de proporción de urbano y rural}
\protect\hypertarget{estimaciuxf3n-de-proporciuxf3n-de-urbano-y-rural}{}
El procedimiento estándar para el calculo de proporciones es crear una
\emph{variable dummy} y sobre está realizar las operaciones. Sin
embargo, la librería \texttt{srvy} nos simplifica el calculo, mediante
la sintaxis.

\begin{Shaded}
\begin{Highlighting}[]
\NormalTok{(prop\_zona }\OtherTok{\textless{}{-}}\NormalTok{ diseno }\SpecialCharTok{\%\textgreater{}\%} \FunctionTok{group\_by}\NormalTok{(Zone) }\SpecialCharTok{\%\textgreater{}\%} 
   \FunctionTok{summarise}\NormalTok{(}
     \AttributeTok{prop =} \FunctionTok{survey\_mean}\NormalTok{(}\AttributeTok{vartype =} \FunctionTok{c}\NormalTok{(}\StringTok{"se"}\NormalTok{,}\StringTok{"ci"}\NormalTok{), }
                        \AttributeTok{proportion =} \ConstantTok{TRUE}\NormalTok{ )))}
\end{Highlighting}
\end{Shaded}

\begin{longtable}[]{@{}lrrrr@{}}
\toprule
Zone & prop & prop\_se & prop\_low & prop\_upp \\
\midrule
\endhead
Rural & 0.4798 & 0.014 & 0.4523 & 0.5075 \\
Urban & 0.5202 & 0.014 & 0.4925 & 0.5477 \\
\bottomrule
\end{longtable}

Note que, se utilzo la función \texttt{survey\_mean} para la estimación.
\end{frame}

\begin{frame}[fragile]{Estimación de proporción de urbano y rural}
\protect\hypertarget{estimaciuxf3n-de-proporciuxf3n-de-urbano-y-rural-1}{}
La función idónea para realizar la estimación de las proporciones es
\texttt{survey\_prop} y la sintaxis es como sigue:

\begin{Shaded}
\begin{Highlighting}[]
\NormalTok{(prop\_zona2 }\OtherTok{\textless{}{-}}\NormalTok{ diseno }\SpecialCharTok{\%\textgreater{}\%} \FunctionTok{group\_by}\NormalTok{(Zone) }\SpecialCharTok{\%\textgreater{}\%} 
   \FunctionTok{summarise}\NormalTok{(}
     \AttributeTok{prop =} \FunctionTok{survey\_prop}\NormalTok{(}\AttributeTok{vartype =} \FunctionTok{c}\NormalTok{(}\StringTok{"se"}\NormalTok{,}\StringTok{"ci"}\NormalTok{) )))}
\end{Highlighting}
\end{Shaded}

\begin{longtable}[]{@{}lrrrr@{}}
\toprule
Zone & prop & prop\_se & prop\_low & prop\_upp \\
\midrule
\endhead
Rural & 0.4798 & 0.014 & 0.4522 & 0.5075 \\
Urban & 0.5202 & 0.014 & 0.4925 & 0.5478 \\
\bottomrule
\end{longtable}
\end{frame}

\begin{frame}[fragile]{Proporción de hombres y mujeres en la zona
urbana}
\protect\hypertarget{proporciuxf3n-de-hombres-y-mujeres-en-la-zona-urbana}{}
Si el interés es obtener la estimación para una subpoblación, procedemos
así:

\begin{Shaded}
\begin{Highlighting}[]
\NormalTok{(prop\_sexoU }\OtherTok{\textless{}{-}}\NormalTok{ sub\_Urbano }\SpecialCharTok{\%\textgreater{}\%} \FunctionTok{group\_by}\NormalTok{(Sex) }\SpecialCharTok{\%\textgreater{}\%} 
   \FunctionTok{summarise}\NormalTok{(}
       \AttributeTok{prop =} \FunctionTok{survey\_prop}\NormalTok{(}\AttributeTok{vartype =} \FunctionTok{c}\NormalTok{(}\StringTok{"se"}\NormalTok{,}\StringTok{"ci"}\NormalTok{))))}
\end{Highlighting}
\end{Shaded}

\begin{longtable}[]{@{}lrrrr@{}}
\toprule
Sex & prop & prop\_se & prop\_low & prop\_upp \\
\midrule
\endhead
Female & 0.5367 & 0.013 & 0.5108 & 0.5627 \\
Male & 0.4633 & 0.013 & 0.4373 & 0.4892 \\
\bottomrule
\end{longtable}

\pause

\begin{block}{Ejercicio}
\protect\hypertarget{ejercicio}{}
¿Cómo estimar el Proporción de hombres dado que están en zona rural?
\end{block}
\end{frame}

\begin{frame}[fragile]{Proporción de hombres y mujeres en la zona rural}
\protect\hypertarget{proporciuxf3n-de-hombres-y-mujeres-en-la-zona-rural}{}
\begin{Shaded}
\begin{Highlighting}[]
\NormalTok{(prop\_sexoR }\OtherTok{\textless{}{-}}\NormalTok{ sub\_Rural }\SpecialCharTok{\%\textgreater{}\%} \FunctionTok{group\_by}\NormalTok{(Sex) }\SpecialCharTok{\%\textgreater{}\%} 
   \FunctionTok{summarise}\NormalTok{(}
     \AttributeTok{n =} \FunctionTok{unweighted}\NormalTok{(}\FunctionTok{n}\NormalTok{()),}
     \AttributeTok{prop =} \FunctionTok{survey\_prop}\NormalTok{(}\AttributeTok{vartype =} \FunctionTok{c}\NormalTok{(}\StringTok{"se"}\NormalTok{,}\StringTok{"ci"}\NormalTok{))))}
\end{Highlighting}
\end{Shaded}

\begin{longtable}[]{@{}lrrrrr@{}}
\toprule
Sex & n & prop & prop\_se & prop\_low & prop\_upp \\
\midrule
\endhead
Female & 679 & 0.5165 & 0.0082 & 0.500 & 0.533 \\
Male & 618 & 0.4835 & 0.0082 & 0.467 & 0.500 \\
\bottomrule
\end{longtable}

\pause

\begin{block}{Ejercicio}
\protect\hypertarget{ejercicio-1}{}
¿Cómo estimar el Proporción de hombres en la zona rural dado que es
hombre?
\end{block}
\end{frame}

\begin{frame}[fragile]{Proporción de hombres en la zona urbana y rural}
\protect\hypertarget{proporciuxf3n-de-hombres-en-la-zona-urbana-y-rural}{}
\begin{Shaded}
\begin{Highlighting}[]
\NormalTok{(prop\_ZonaH }\OtherTok{\textless{}{-}}\NormalTok{ sub\_Hombre }\SpecialCharTok{\%\textgreater{}\%} \FunctionTok{group\_by}\NormalTok{(Zone) }\SpecialCharTok{\%\textgreater{}\%} 
   \FunctionTok{summarise}\NormalTok{(}
     \AttributeTok{prop =} \FunctionTok{survey\_prop}\NormalTok{(}\AttributeTok{vartype =} \FunctionTok{c}\NormalTok{(}\StringTok{"se"}\NormalTok{,}\StringTok{"ci"}\NormalTok{))))}
\end{Highlighting}
\end{Shaded}

\begin{longtable}[]{@{}lrrrr@{}}
\toprule
Zone & prop & prop\_se & prop\_low & prop\_upp \\
\midrule
\endhead
Rural & 0.4905 & 0.0178 & 0.4552 & 0.5258 \\
Urban & 0.5095 & 0.0178 & 0.4742 & 0.5448 \\
\bottomrule
\end{longtable}

\pause

\begin{block}{Ejercicio}
\protect\hypertarget{ejercicio-2}{}
¿Cómo estimar el Proporción de mujeres en la zona rural dado que es
mujer?
\end{block}
\end{frame}

\begin{frame}[fragile]{Proporción de mujeres en la zona urbana y rural}
\protect\hypertarget{proporciuxf3n-de-mujeres-en-la-zona-urbana-y-rural}{}
\begin{Shaded}
\begin{Highlighting}[]
\NormalTok{(prop\_ZonaM }\OtherTok{\textless{}{-}}\NormalTok{ sub\_Mujer }\SpecialCharTok{\%\textgreater{}\%} \FunctionTok{group\_by}\NormalTok{(Zone) }\SpecialCharTok{\%\textgreater{}\%} 
   \FunctionTok{summarise}\NormalTok{(}
    \AttributeTok{prop =} \FunctionTok{survey\_prop}\NormalTok{(}\AttributeTok{vartype =} \FunctionTok{c}\NormalTok{(}\StringTok{"se"}\NormalTok{,}\StringTok{"ci"}\NormalTok{))))}
\end{Highlighting}
\end{Shaded}

\begin{longtable}[]{@{}lrrrr@{}}
\toprule
Zone & prop & prop\_se & prop\_low & prop\_upp \\
\midrule
\endhead
Rural & 0.4702 & 0.014 & 0.4425 & 0.4980 \\
Urban & 0.5298 & 0.014 & 0.5020 & 0.5575 \\
\bottomrule
\end{longtable}
\end{frame}

\begin{frame}[fragile]{Proporción de hombres en la zona urbana y rural}
\protect\hypertarget{proporciuxf3n-de-hombres-en-la-zona-urbana-y-rural-1}{}
Con el uso de la función \texttt{group\_by} es posible estimar un mayor
numero de niveles de agregación al combinar dos o más variables.

\begin{Shaded}
\begin{Highlighting}[]
\NormalTok{(prop\_ZonaH\_Pobreza }\OtherTok{\textless{}{-}}\NormalTok{ sub\_Hombre }\SpecialCharTok{\%\textgreater{}\%}
   \FunctionTok{group\_by}\NormalTok{(Zone, Poverty) }\SpecialCharTok{\%\textgreater{}\%} 
   \FunctionTok{summarise}\NormalTok{(}
     \AttributeTok{prop =} \FunctionTok{survey\_prop}\NormalTok{(}\AttributeTok{vartype =} \FunctionTok{c}\NormalTok{(}\StringTok{"se"}\NormalTok{,}\StringTok{"ci"}\NormalTok{)))}\SpecialCharTok{\%\textgreater{}\%}
   \FunctionTok{data.frame}\NormalTok{())}
\end{Highlighting}
\end{Shaded}
\end{frame}

\begin{frame}{Proporción de hombres en la zona urbana y rural}
\protect\hypertarget{proporciuxf3n-de-hombres-en-la-zona-urbana-y-rural-2}{}
\begin{longtable}[]{@{}llrrrr@{}}
\toprule
Zone & Poverty & prop & prop\_se & prop\_low & prop\_upp \\
\midrule
\endhead
Rural & NotPoor & 0.5488 & 0.0626 & 0.4248 & 0.6729 \\
Rural & Extreme & 0.1975 & 0.0675 & 0.0640 & 0.3311 \\
Rural & Relative & 0.2536 & 0.0372 & 0.1799 & 0.3274 \\
Urban & NotPoor & 0.6599 & 0.0366 & 0.5874 & 0.7324 \\
Urban & Extreme & 0.1129 & 0.0245 & 0.0643 & 0.1614 \\
Urban & Relative & 0.2272 & 0.0260 & 0.1757 & 0.2788 \\
\bottomrule
\end{longtable}
\end{frame}

\begin{frame}[fragile]{Proporción de mujeres en la zona urbana y rural}
\protect\hypertarget{proporciuxf3n-de-mujeres-en-la-zona-urbana-y-rural-1}{}
\begin{Shaded}
\begin{Highlighting}[]
\NormalTok{(prop\_ZonaM\_Pobreza }\OtherTok{\textless{}{-}}\NormalTok{ sub\_Mujer }\SpecialCharTok{\%\textgreater{}\%} 
   \FunctionTok{group\_by}\NormalTok{(Zone, Poverty) }\SpecialCharTok{\%\textgreater{}\%} 
   \FunctionTok{summarise}\NormalTok{(}
     \AttributeTok{prop =} \FunctionTok{survey\_prop}\NormalTok{(}\AttributeTok{vartype =} \FunctionTok{c}\NormalTok{(}\StringTok{"se"}\NormalTok{,}\StringTok{"ci"}\NormalTok{))) }\SpecialCharTok{\%\textgreater{}\%}
   \FunctionTok{data.frame}\NormalTok{())}
\end{Highlighting}
\end{Shaded}

\begin{longtable}[]{@{}llrrrr@{}}
\toprule
Zone & Poverty & prop & prop\_se & prop\_low & prop\_upp \\
\midrule
\endhead
Rural & NotPoor & 0.5539 & 0.0557 & 0.4436 & 0.6642 \\
Rural & Extreme & 0.1600 & 0.0557 & 0.0496 & 0.2704 \\
Rural & Relative & 0.2861 & 0.0436 & 0.1998 & 0.3724 \\
Urban & NotPoor & 0.6612 & 0.0322 & 0.5974 & 0.7251 \\
Urban & Extreme & 0.1094 & 0.0221 & 0.0656 & 0.1531 \\
Urban & Relative & 0.2294 & 0.0266 & 0.1768 & 0.2820 \\
\bottomrule
\end{longtable}
\end{frame}

\begin{frame}[fragile]{Proporción de hombres en la zona y empleado}
\protect\hypertarget{proporciuxf3n-de-hombres-en-la-zona-y-empleado}{}
\begin{Shaded}
\begin{Highlighting}[]
\NormalTok{(prop\_ZonaH\_Ocupacion }\OtherTok{\textless{}{-}}\NormalTok{ sub\_Hombre }\SpecialCharTok{\%\textgreater{}\%}
   \FunctionTok{group\_by}\NormalTok{(Zone, Employment) }\SpecialCharTok{\%\textgreater{}\%} 
   \FunctionTok{summarise}\NormalTok{(}
     \AttributeTok{prop =} \FunctionTok{survey\_prop}\NormalTok{(}\AttributeTok{vartype =} \FunctionTok{c}\NormalTok{(}\StringTok{"se"}\NormalTok{,}\StringTok{"ci"}\NormalTok{)))}\SpecialCharTok{\%\textgreater{}\%}
   \FunctionTok{data.frame}\NormalTok{())}
\end{Highlighting}
\end{Shaded}

\begin{longtable}[]{@{}llrrrr@{}}
\toprule
Zone & Employment & prop & prop\_se & prop\_low & prop\_upp \\
\midrule
\endhead
Rural & Unemployed & 0.0513 & 0.0157 & 0.0201 & 0.0824 \\
Rural & Inactive & 0.1035 & 0.0203 & 0.0634 & 0.1436 \\
Rural & Employed & 0.5225 & 0.0265 & 0.4700 & 0.5750 \\
Rural & NA & 0.3227 & 0.0350 & 0.2534 & 0.3920 \\
Urban & Unemployed & 0.0437 & 0.0085 & 0.0269 & 0.0606 \\
Urban & Inactive & 0.1633 & 0.0181 & 0.1275 & 0.1991 \\
Urban & Employed & 0.5134 & 0.0236 & 0.4666 & 0.5602 \\
Urban & NA & 0.2796 & 0.0221 & 0.2358 & 0.3233 \\
\bottomrule
\end{longtable}
\end{frame}

\begin{frame}[fragile]{Proporción de mujeres en la zona urbana y rural}
\protect\hypertarget{proporciuxf3n-de-mujeres-en-la-zona-urbana-y-rural-2}{}
\begin{Shaded}
\begin{Highlighting}[]
\NormalTok{(prop\_ZonaM\_Ocupacion }\OtherTok{\textless{}{-}}\NormalTok{ sub\_Mujer }\SpecialCharTok{\%\textgreater{}\%} 
   \FunctionTok{group\_by}\NormalTok{(Zone, Employment) }\SpecialCharTok{\%\textgreater{}\%} 
   \FunctionTok{summarise}\NormalTok{(}
     \AttributeTok{prop =} \FunctionTok{survey\_prop}\NormalTok{(}
       \AttributeTok{vartype =} \FunctionTok{c}\NormalTok{(}\StringTok{"se"}\NormalTok{,}\StringTok{"ci"}\NormalTok{))) }\SpecialCharTok{\%\textgreater{}\%}
   \FunctionTok{data.frame}\NormalTok{())}
\end{Highlighting}
\end{Shaded}

\tiny

\begin{longtable}[]{@{}llrrrr@{}}
\toprule
Zone & Employment & prop & prop\_se & prop\_low & prop\_upp \\
\midrule
\endhead
Rural & Unemployed & 0.0102 & 0.0055 & -0.0008 & 0.0211 \\
Rural & Inactive & 0.4472 & 0.0352 & 0.3774 & 0.5170 \\
Rural & Employed & 0.2400 & 0.0392 & 0.1625 & 0.3175 \\
Rural & NA & 0.3026 & 0.0308 & 0.2417 & 0.3636 \\
Urban & Unemployed & 0.0211 & 0.0060 & 0.0093 & 0.0329 \\
Urban & Inactive & 0.3645 & 0.0214 & 0.3220 & 0.4069 \\
Urban & Employed & 0.3846 & 0.0195 & 0.3460 & 0.4231 \\
Urban & NA & 0.2299 & 0.0139 & 0.2025 & 0.2573 \\
\bottomrule
\end{longtable}
\end{frame}

\begin{frame}[fragile]{Estimación de la proporción de personas menor a
18 años}
\protect\hypertarget{estimaciuxf3n-de-la-proporciuxf3n-de-personas-menor-a-18-auxf1os}{}
\begin{Shaded}
\begin{Highlighting}[]
\NormalTok{diseno }\SpecialCharTok{\%\textgreater{}\%} 
\FunctionTok{group\_by}\NormalTok{(edad\_18, pobreza) }\SpecialCharTok{\%\textgreater{}\%} 
  \FunctionTok{summarise}\NormalTok{(}
    \AttributeTok{Prop =} \FunctionTok{survey\_prop}\NormalTok{(}
      \AttributeTok{vartype =}  \FunctionTok{c}\NormalTok{(}\StringTok{"se"}\NormalTok{, }\StringTok{"ci"}\NormalTok{))) }\SpecialCharTok{\%\textgreater{}\%}
  \FunctionTok{data.frame}\NormalTok{()}
\end{Highlighting}
\end{Shaded}
\end{frame}

\begin{frame}{Estimación de la proporción de personas menor a 18 años}
\protect\hypertarget{estimaciuxf3n-de-la-proporciuxf3n-de-personas-menor-a-18-auxf1os-1}{}
\scriptsize

\begin{longtable}[]{@{}lrrrrr@{}}
\toprule
edad\_18 & pobreza & Prop & Prop\_se & Prop\_low & Prop\_upp \\
\midrule
\endhead
\textless{} 18 anios & 0 & 0.4985 & 0.0373 & 0.4246 & 0.5723 \\
\textless{} 18 anios & 1 & 0.5015 & 0.0373 & 0.4277 & 0.5754 \\
\textgreater= 18 anios & 0 & 0.6646 & 0.0298 & 0.6056 & 0.7236 \\
\textgreater= 18 anios & 1 & 0.3354 & 0.0298 & 0.2764 & 0.3944 \\
\bottomrule
\end{longtable}
\end{frame}

\begin{frame}[fragile]{Estimación de la proporción de personas menor a
18 años}
\protect\hypertarget{estimaciuxf3n-de-la-proporciuxf3n-de-personas-menor-a-18-auxf1os-2}{}
\begin{Shaded}
\begin{Highlighting}[]
\NormalTok{diseno }\SpecialCharTok{\%\textgreater{}\%} 
  \FunctionTok{group\_by}\NormalTok{(edad\_18, desempleo) }\SpecialCharTok{\%\textgreater{}\%} 
  \FunctionTok{summarise}\NormalTok{(}
    \AttributeTok{Prop =} \FunctionTok{survey\_prop}\NormalTok{(}
      \AttributeTok{vartype =}  \FunctionTok{c}\NormalTok{(}\StringTok{"se"}\NormalTok{, }\StringTok{"ci"}\NormalTok{))) }\SpecialCharTok{\%\textgreater{}\%}
  \FunctionTok{data.frame}\NormalTok{()}
\end{Highlighting}
\end{Shaded}
\end{frame}

\begin{frame}{Estimación de la proporción de personas menor a 18 años}
\protect\hypertarget{estimaciuxf3n-de-la-proporciuxf3n-de-personas-menor-a-18-auxf1os-3}{}
\scriptsize

\begin{longtable}[]{@{}lrrrrr@{}}
\toprule
edad\_18 & desempleo & Prop & Prop\_se & Prop\_low & Prop\_upp \\
\midrule
\endhead
\textless{} 18 anios & 0 & 0.1667 & 0.0149 & 0.1373 & 0.1961 \\
\textless{} 18 anios & 1 & 0.0037 & 0.0020 & -0.0002 & 0.0076 \\
\textless{} 18 anios & NA & 0.8296 & 0.0150 & 0.7998 & 0.8593 \\
\textgreater= 18 anios & 0 & 0.9552 & 0.0076 & 0.9403 & 0.9702 \\
\textgreater= 18 anios & 1 & 0.0448 & 0.0076 & 0.0298 & 0.0597 \\
\bottomrule
\end{longtable}
\end{frame}

\begin{frame}[fragile]{Estimación de la proporción de personas menor a
18 años en zona rural}
\protect\hypertarget{estimaciuxf3n-de-la-proporciuxf3n-de-personas-menor-a-18-auxf1os-en-zona-rural}{}
\begin{Shaded}
\begin{Highlighting}[]
\NormalTok{sub\_Rural }\SpecialCharTok{\%\textgreater{}\%}
  \FunctionTok{group\_by}\NormalTok{(edad\_18) }\SpecialCharTok{\%\textgreater{}\%} 
  \FunctionTok{summarise}\NormalTok{(}
    \AttributeTok{Prop =} \FunctionTok{survey\_prop}\NormalTok{(}
      \AttributeTok{vartype =}  \FunctionTok{c}\NormalTok{(}\StringTok{"se"}\NormalTok{, }\StringTok{"ci"}\NormalTok{))) }\SpecialCharTok{\%\textgreater{}\%}
  \FunctionTok{data.frame}\NormalTok{()}
\end{Highlighting}
\end{Shaded}

\begin{longtable}[]{@{}lrrrr@{}}
\toprule
edad\_18 & Prop & Prop\_se & Prop\_low & Prop\_upp \\
\midrule
\endhead
\textless{} 18 anios & 0.3712 & 0.0302 & 0.3106 & 0.4317 \\
\textgreater= 18 anios & 0.6288 & 0.0302 & 0.5683 & 0.6894 \\
\bottomrule
\end{longtable}
\end{frame}

\begin{frame}[fragile]{Estimación de la proporción de mujeres rango de
edad}
\protect\hypertarget{estimaciuxf3n-de-la-proporciuxf3n-de-mujeres-rango-de-edad}{}
\begin{Shaded}
\begin{Highlighting}[]
\NormalTok{sub\_Mujer }\SpecialCharTok{\%\textgreater{}\%} \FunctionTok{mutate}\NormalTok{(}\AttributeTok{edad\_rango =} \FunctionTok{case\_when}\NormalTok{(}
\NormalTok{  Age}\SpecialCharTok{\textgreater{}=} \DecValTok{18} \SpecialCharTok{\&}\NormalTok{ Age }\SpecialCharTok{\textless{}=}\DecValTok{35}  \SpecialCharTok{\textasciitilde{}} \StringTok{"18 {-} 35"}\NormalTok{,}
  \ConstantTok{TRUE} \SpecialCharTok{\textasciitilde{}} \StringTok{"Otro"}\NormalTok{)) }\SpecialCharTok{\%\textgreater{}\%}
  \FunctionTok{group\_by}\NormalTok{(edad\_rango, Employment) }\SpecialCharTok{\%\textgreater{}\%} 
  \FunctionTok{summarise}\NormalTok{(}
    \AttributeTok{Prop =} \FunctionTok{survey\_prop}\NormalTok{(}
      \AttributeTok{vartype =}  \FunctionTok{c}\NormalTok{(}\StringTok{"se"}\NormalTok{, }\StringTok{"ci"}\NormalTok{))) }\SpecialCharTok{\%\textgreater{}\%}
  \FunctionTok{data.frame}\NormalTok{()}
\end{Highlighting}
\end{Shaded}

\scriptsize

\begin{longtable}[]{@{}llrrrr@{}}
\toprule
edad\_rango & Employment & Prop & Prop\_se & Prop\_low & Prop\_upp \\
\midrule
\endhead
18 - 35 & Unemployed & 0.0289 & 0.0091 & 0.0108 & 0.0470 \\
18 - 35 & Inactive & 0.5165 & 0.0379 & 0.4415 & 0.5916 \\
18 - 35 & Employed & 0.4545 & 0.0357 & 0.3839 & 0.5252 \\
Otro & Unemployed & 0.0102 & 0.0040 & 0.0022 & 0.0181 \\
Otro & Inactive & 0.3527 & 0.0207 & 0.3117 & 0.3937 \\
Otro & Employed & 0.2548 & 0.0217 & 0.2119 & 0.2978 \\
Otro & NA & 0.3823 & 0.0223 & 0.3381 & 0.4265 \\
\bottomrule
\end{longtable}
\end{frame}

\begin{frame}[fragile]{Estimación de la proporción de hombres rango de
edad}
\protect\hypertarget{estimaciuxf3n-de-la-proporciuxf3n-de-hombres-rango-de-edad}{}
\begin{Shaded}
\begin{Highlighting}[]
\NormalTok{sub\_Hombre }\SpecialCharTok{\%\textgreater{}\%} \FunctionTok{mutate}\NormalTok{(}\AttributeTok{edad\_rango =} \FunctionTok{case\_when}\NormalTok{(}
\NormalTok{  Age}\SpecialCharTok{\textgreater{}=} \DecValTok{18} \SpecialCharTok{\&}\NormalTok{ Age }\SpecialCharTok{\textless{}=}\DecValTok{35}  \SpecialCharTok{\textasciitilde{}} \StringTok{"18 {-} 35"}\NormalTok{,}
  \ConstantTok{TRUE} \SpecialCharTok{\textasciitilde{}} \StringTok{"Otro"}\NormalTok{)) }\SpecialCharTok{\%\textgreater{}\%}
  \FunctionTok{group\_by}\NormalTok{(edad\_rango, Employment) }\SpecialCharTok{\%\textgreater{}\%} 
  \FunctionTok{summarise}\NormalTok{(}
    \AttributeTok{Prop =} \FunctionTok{survey\_prop}\NormalTok{(}
      \AttributeTok{vartype =}  \FunctionTok{c}\NormalTok{(}\StringTok{"se"}\NormalTok{, }\StringTok{"ci"}\NormalTok{))) }\SpecialCharTok{\%\textgreater{}\%}
  \FunctionTok{data.frame}\NormalTok{()}
\end{Highlighting}
\end{Shaded}

\scriptsize

\begin{longtable}[]{@{}llrrrr@{}}
\toprule
edad\_rango & Employment & Prop & Prop\_se & Prop\_low & Prop\_upp \\
\midrule
\endhead
18 - 35 & Unemployed & 0.0964 & 0.0182 & 0.0603 & 0.1324 \\
18 - 35 & Inactive & 0.0894 & 0.0164 & 0.0568 & 0.1219 \\
18 - 35 & Employed & 0.8142 & 0.0230 & 0.7687 & 0.8598 \\
Otro & Unemployed & 0.0261 & 0.0072 & 0.0119 & 0.0403 \\
Otro & Inactive & 0.1534 & 0.0199 & 0.1141 & 0.1928 \\
Otro & Employed & 0.3885 & 0.0203 & 0.3484 & 0.4286 \\
Otro & NA & 0.4320 & 0.0211 & 0.3902 & 0.4738 \\
\bottomrule
\end{longtable}
\end{frame}

\hypertarget{tablas-cruzadas.}{%
\section{Tablas cruzadas.}\label{tablas-cruzadas.}}

\begin{frame}[fragile]{Tabla Zona Vs Sexo}
\protect\hypertarget{tabla-zona-vs-sexo}{}
Haciendo uso de la función \texttt{group\_by} organizada en forma de
\texttt{data.frame}.

\begin{Shaded}
\begin{Highlighting}[]
\NormalTok{(}
\NormalTok{  prop\_sexo\_zona }\OtherTok{\textless{}{-}}\NormalTok{ diseno }\SpecialCharTok{\%\textgreater{}\%} 
    \FunctionTok{group\_by}\NormalTok{(pobreza,Sex) }\SpecialCharTok{\%\textgreater{}\%}
    \FunctionTok{summarise}\NormalTok{(}
      \AttributeTok{prop =} \FunctionTok{survey\_prop}\NormalTok{(}\AttributeTok{vartype =} \FunctionTok{c}\NormalTok{(}\StringTok{"se"}\NormalTok{, }\StringTok{"ci"}\NormalTok{))) }\SpecialCharTok{\%\textgreater{}\%} 
    \FunctionTok{data.frame}\NormalTok{()}
\NormalTok{)}
\end{Highlighting}
\end{Shaded}

Esta forma de organizar la información es recomendable cuando el
realizar el análisis sobre las estimaciones puntuales.
\end{frame}

\begin{frame}{Tabla Zona Vs Sexo}
\protect\hypertarget{tabla-zona-vs-sexo-1}{}
\scriptsize

\begin{longtable}[]{@{}rlrrrr@{}}
\toprule
pobreza & Sex & prop & prop\_se & prop\_low & prop\_upp \\
\midrule
\endhead
0 & Female & 0.5292 & 0.0124 & 0.5046 & 0.5538 \\
0 & Male & 0.4708 & 0.0124 & 0.4462 & 0.4954 \\
1 & Female & 0.5236 & 0.0159 & 0.4922 & 0.5550 \\
1 & Male & 0.4764 & 0.0159 & 0.4450 & 0.5078 \\
\bottomrule
\end{longtable}
\end{frame}

\begin{frame}[fragile]{Tablas de doble entrada.}
\protect\hypertarget{tablas-de-doble-entrada.}{}
Una alternativa es utilizar la función \texttt{svyby} con la siguiente
sintaxis. \scriptsize

\begin{Shaded}
\begin{Highlighting}[]
\NormalTok{tab\_Sex\_Pobr }\OtherTok{\textless{}{-}} \FunctionTok{svyby}\NormalTok{(}\SpecialCharTok{\textasciitilde{}}\NormalTok{Sex,  }\SpecialCharTok{\textasciitilde{}}\NormalTok{pobreza, diseno, svymean)}
\NormalTok{tab\_Sex\_Pobr }\SpecialCharTok{\%\textgreater{}\%} \FunctionTok{select}\NormalTok{(}\SpecialCharTok{{-}}\NormalTok{se.SexFemale, }\SpecialCharTok{{-}}\NormalTok{se.SexMale)}
\end{Highlighting}
\end{Shaded}

\begin{longtable}[]{@{}lrrr@{}}
\toprule
& pobreza & SexFemale & SexMale \\
\midrule
\endhead
0 & 0 & 0.5292 & 0.4708 \\
1 & 1 & 0.5236 & 0.4764 \\
\bottomrule
\end{longtable}

\begin{Shaded}
\begin{Highlighting}[]
\NormalTok{tab\_Sex\_Pobr }\SpecialCharTok{\%\textgreater{}\%} \FunctionTok{select}\NormalTok{(}\SpecialCharTok{{-}}\NormalTok{SexFemale, }\SpecialCharTok{{-}}\NormalTok{SexMale)}
\end{Highlighting}
\end{Shaded}

\begin{longtable}[]{@{}lrrr@{}}
\toprule
& pobreza & se.SexFemale & se.SexMale \\
\midrule
\endhead
0 & 0 & 0.0124 & 0.0124 \\
1 & 1 & 0.0159 & 0.0159 \\
\bottomrule
\end{longtable}
\end{frame}

\begin{frame}[fragile]{Tablas de doble entrada.}
\protect\hypertarget{tablas-de-doble-entrada.-1}{}
Para la estimación de los intervalos de confianza utilizar la función
\texttt{confint}.

\begin{Shaded}
\begin{Highlighting}[]
\FunctionTok{confint}\NormalTok{(tab\_Sex\_Pobr) }\SpecialCharTok{\%\textgreater{}\%} \FunctionTok{as.data.frame}\NormalTok{()}
\end{Highlighting}
\end{Shaded}

\begin{longtable}[]{@{}lrr@{}}
\toprule
& 2.5 \% & 97.5 \% \\
\midrule
\endhead
0:SexFemale & 0.5048 & 0.5535 \\
1:SexFemale & 0.4925 & 0.5547 \\
0:SexMale & 0.4465 & 0.4952 \\
1:SexMale & 0.4453 & 0.5075 \\
\bottomrule
\end{longtable}
\end{frame}

\begin{frame}[fragile]{Prueba de independencia.}
\protect\hypertarget{prueba-de-independencia.}{}
Para realizar la prueba de independencia \(\chi^{2}\) puede ejecuta la
siguiente linea de código.

\begin{Shaded}
\begin{Highlighting}[]
\FunctionTok{svychisq}\NormalTok{(}\SpecialCharTok{\textasciitilde{}}\NormalTok{Sex }\SpecialCharTok{+}\NormalTok{ pobreza, diseno, }\AttributeTok{statistic=}\StringTok{"F"}\NormalTok{)}
\end{Highlighting}
\end{Shaded}

\begin{verbatim}
## 
##  Pearson's X^2: Rao & Scott adjustment
## 
## data:  NextMethod()
## F = 0.056, ndf = 1, ddf = 119, p-value = 0.8
\end{verbatim}

Más adelante se profundiza en la metodología de esta prueba.
\end{frame}

\begin{frame}[fragile]{Tablas de doble entrada.}
\protect\hypertarget{tablas-de-doble-entrada.-2}{}
\begin{Shaded}
\begin{Highlighting}[]
\NormalTok{(tab\_Sex\_Ocupa }\OtherTok{\textless{}{-}} \FunctionTok{svyby}\NormalTok{(}\SpecialCharTok{\textasciitilde{}}\NormalTok{Sex,  }\SpecialCharTok{\textasciitilde{}}\NormalTok{Employment,}
\NormalTok{                       diseno, svymean))}
\end{Highlighting}
\end{Shaded}

\scriptsize

\begin{longtable}[]{@{}llrrrr@{}}
\toprule
& Employment & SexFemale & SexMale & se.SexFemale & se.SexMale \\
\midrule
\endhead
Unemployed & Unemployed & 0.2727 & 0.7273 & 0.0535 & 0.0535 \\
Inactive & Inactive & 0.7703 & 0.2297 & 0.0234 & 0.0234 \\
Employed & Employed & 0.4052 & 0.5948 & 0.0185 & 0.0185 \\
\bottomrule
\end{longtable}
\end{frame}

\begin{frame}[fragile]{Tablas de doble entrada}
\protect\hypertarget{tablas-de-doble-entrada}{}
\begin{Shaded}
\begin{Highlighting}[]
\FunctionTok{confint}\NormalTok{(tab\_Sex\_Ocupa) }\SpecialCharTok{\%\textgreater{}\%} \FunctionTok{as.data.frame}\NormalTok{()}
\end{Highlighting}
\end{Shaded}

\begin{longtable}[]{@{}lrr@{}}
\toprule
& 2.5 \% & 97.5 \% \\
\midrule
\endhead
Unemployed:SexFemale & 0.1678 & 0.3776 \\
Inactive:SexFemale & 0.7245 & 0.8162 \\
Employed:SexFemale & 0.3689 & 0.4415 \\
Unemployed:SexMale & 0.6224 & 0.8322 \\
Inactive:SexMale & 0.1838 & 0.2755 \\
Employed:SexMale & 0.5585 & 0.6311 \\
\bottomrule
\end{longtable}
\end{frame}

\begin{frame}[fragile]{Prueba de independencia.}
\protect\hypertarget{prueba-de-independencia.-1}{}
La prueba de independencia \(\chi^{2}\) se obtiene con la siguiente
linea de código.

\begin{Shaded}
\begin{Highlighting}[]
\FunctionTok{svychisq}\NormalTok{(}\SpecialCharTok{\textasciitilde{}}\NormalTok{Sex }\SpecialCharTok{+}\NormalTok{ Employment, }
         \AttributeTok{design =}\NormalTok{ diseno,  }\AttributeTok{statistic=}\StringTok{"F"}\NormalTok{)}
\end{Highlighting}
\end{Shaded}

\begin{verbatim}
## 
##  Pearson's X^2: Rao & Scott adjustment
## 
## data:  NextMethod()
## F = 62, ndf = 1.7, ddf = 200.7, p-value <2e-16
\end{verbatim}
\end{frame}

\begin{frame}[fragile]{Tablas de doble entrada.}
\protect\hypertarget{tablas-de-doble-entrada.-3}{}
Dado que la variable \emph{pobreza} es de tipo númerica, es necesario
convertirla en factor. \scriptsize

\begin{Shaded}
\begin{Highlighting}[]
\NormalTok{tab\_region\_pobreza }\OtherTok{\textless{}{-}} 
  \FunctionTok{svyby}\NormalTok{(}\SpecialCharTok{\textasciitilde{}}\FunctionTok{as.factor}\NormalTok{(pobreza),  }\SpecialCharTok{\textasciitilde{}}\NormalTok{Region, diseno, svymean)}
\NormalTok{tab\_region\_pobreza }\SpecialCharTok{\%\textgreater{}\%} 
  \FunctionTok{select}\NormalTok{(}\SpecialCharTok{{-}}\StringTok{"se.as.factor(pobreza)0"}\NormalTok{,}
         \SpecialCharTok{{-}}\StringTok{"se.as.factor(pobreza)1"}\NormalTok{)}
\end{Highlighting}
\end{Shaded}

\begin{longtable}[]{@{}llrr@{}}
\toprule
& Region & as.factor(pobreza)0 & as.factor(pobreza)1 \\
\midrule
\endhead
Norte & Norte & 0.6410 & 0.3590 \\
Sur & Sur & 0.6562 & 0.3438 \\
Centro & Centro & 0.6346 & 0.3654 \\
Occidente & Occidente & 0.5992 & 0.4008 \\
Oriente & Oriente & 0.5482 & 0.4518 \\
\bottomrule
\end{longtable}
\end{frame}

\begin{frame}[fragile]{Tablas de doble entrada.}
\protect\hypertarget{tablas-de-doble-entrada.-4}{}
\begin{Shaded}
\begin{Highlighting}[]
\NormalTok{tab\_region\_pobreza }\SpecialCharTok{\%\textgreater{}\%} 
 \FunctionTok{select}\NormalTok{(}\StringTok{"se.as.factor(pobreza)0"}\NormalTok{,}
        \StringTok{"se.as.factor(pobreza)1"}\NormalTok{)}
\end{Highlighting}
\end{Shaded}

\begin{longtable}[]{@{}lrr@{}}
\toprule
& se.as.factor(pobreza)0 & se.as.factor(pobreza)1 \\
\midrule
\endhead
Norte & 0.0555 & 0.0555 \\
Sur & 0.0435 & 0.0435 \\
Centro & 0.0786 & 0.0786 \\
Occidente & 0.0467 & 0.0467 \\
Oriente & 0.0885 & 0.0885 \\
\bottomrule
\end{longtable}
\end{frame}

\begin{frame}[fragile]{Prueba de independencia.}
\protect\hypertarget{prueba-de-independencia.-2}{}
Una vez más la prueba de independencia es:

\begin{Shaded}
\begin{Highlighting}[]
\FunctionTok{svychisq}\NormalTok{(}\SpecialCharTok{\textasciitilde{}}\NormalTok{Region }\SpecialCharTok{+}\NormalTok{ pobreza, }
         \AttributeTok{design =}\NormalTok{ diseno,  }\AttributeTok{statistic=}\StringTok{"F"}\NormalTok{)}
\end{Highlighting}
\end{Shaded}

\begin{verbatim}
## 
##  Pearson's X^2: Rao & Scott adjustment
## 
## data:  NextMethod()
## F = 0.49, ndf = 3, ddf = 358, p-value = 0.7
\end{verbatim}
\end{frame}

\begin{frame}[fragile]{Razón de obbs}
\protect\hypertarget{razuxf3n-de-obbs}{}
\begin{Shaded}
\begin{Highlighting}[]
\NormalTok{(tab\_Sex }\OtherTok{\textless{}{-}} \FunctionTok{svyby}\NormalTok{(}\SpecialCharTok{\textasciitilde{}}\NormalTok{pobreza,  }\SpecialCharTok{\textasciitilde{}}\NormalTok{Sex, diseno,}
\NormalTok{                 svymean, }\AttributeTok{vartype =} \FunctionTok{c}\NormalTok{(}\StringTok{"se"}\NormalTok{, }\StringTok{"ci"}\NormalTok{)))}
\end{Highlighting}
\end{Shaded}

\begin{longtable}[]{@{}llrrrr@{}}
\toprule
& Sex & pobreza & se & ci\_l & ci\_u \\
\midrule
\endhead
Female & Female & 0.3892 & 0.0316 & 0.3273 & 0.4512 \\
Male & Male & 0.3946 & 0.0366 & 0.3228 & 0.4664 \\
\bottomrule
\end{longtable}

\begin{Shaded}
\begin{Highlighting}[]
\FunctionTok{svycontrast}\NormalTok{(tab\_Sex, }\FunctionTok{quote}\NormalTok{(}\StringTok{\textasciigrave{}}\AttributeTok{Female}\StringTok{\textasciigrave{}}\SpecialCharTok{/}\StringTok{\textasciigrave{}}\AttributeTok{Male}\StringTok{\textasciigrave{}}\NormalTok{)  )}
\end{Highlighting}
\end{Shaded}

\begin{verbatim}
##          nlcon   SE
## contrast 0.987 0.12
\end{verbatim}
\end{frame}

\begin{frame}[fragile]{Razón de obbs}
\protect\hypertarget{razuxf3n-de-obbs-1}{}
\scriptsize

\begin{Shaded}
\begin{Highlighting}[]
\NormalTok{tab\_Sex\_Pobr }\OtherTok{\textless{}{-}} 
   \FunctionTok{svymean}\NormalTok{(}\SpecialCharTok{\textasciitilde{}}\FunctionTok{interaction}\NormalTok{ (Sex, pobreza), diseno, }
             \AttributeTok{se=}\NormalTok{T, }\AttributeTok{na.rm=}\NormalTok{T, }\AttributeTok{ci=}\NormalTok{T, }\AttributeTok{keep.vars=}\NormalTok{T) }
\NormalTok{  tab\_Sex\_Pobr }\SpecialCharTok{\%\textgreater{}\%}  \FunctionTok{as.data.frame}\NormalTok{()}
\end{Highlighting}
\end{Shaded}

\begin{longtable}[]{@{}lrr@{}}
\toprule
& mean & SE \\
\midrule
\endhead
interaction(Sex, pobreza)Female.0 & 0.3219 & 0.0178 \\
interaction(Sex, pobreza)Male.0 & 0.2864 & 0.0177 \\
interaction(Sex, pobreza)Female.1 & 0.2051 & 0.0166 \\
interaction(Sex, pobreza)Male.1 & 0.1866 & 0.0178 \\
\bottomrule
\end{longtable}
\end{frame}

\begin{frame}[fragile]{Razón de obbs}
\protect\hypertarget{razuxf3n-de-obbs-2}{}
Suponga que se desea cálcular la siguiente razón de obbs.

\[
 \frac{\frac{P(Sex = Female \mid pobreza = 0 )}{P(Sex = Female \mid pobreza = 1 )}}{
 \frac{P(Sex = Male \mid pobreza = 1 )}{P(Sex = Male \mid pobreza = 0 )}
 }
\] La forma de cálculo en sería: \footnotesize

\begin{Shaded}
\begin{Highlighting}[]
\FunctionTok{svycontrast}\NormalTok{(tab\_Sex\_Pobr, }
            \FunctionTok{quote}\NormalTok{((}\StringTok{\textasciigrave{}}\AttributeTok{interaction(Sex, pobreza)Female.0}\StringTok{\textasciigrave{}}\SpecialCharTok{/}
                     \StringTok{\textasciigrave{}}\AttributeTok{interaction(Sex, pobreza)Female.1}\StringTok{\textasciigrave{}}\NormalTok{)}\SpecialCharTok{/}
\NormalTok{                  (}\StringTok{\textasciigrave{}}\AttributeTok{interaction(Sex, pobreza)Male.0}\StringTok{\textasciigrave{}}\SpecialCharTok{/}
                     \StringTok{\textasciigrave{}}\AttributeTok{interaction(Sex, pobreza)Male.1}\StringTok{\textasciigrave{}}\NormalTok{) ))}
\end{Highlighting}
\end{Shaded}

\begin{verbatim}
##          nlcon  SE
## contrast  1.02 0.1
\end{verbatim}
\end{frame}

\begin{frame}[fragile]{Razón de obbs}
\protect\hypertarget{razuxf3n-de-obbs-3}{}
Ahora, se desea realizar la siguiente razón: \[
 \frac{\frac{P(Sex = Male \mid pobreza = 1 )}{P(Sex = Female \mid pobreza = 1 )}}{
 \frac{P(Sex = Male \mid pobreza = 0 )}{P(Sex = Female \mid pobreza = 0 )}
 }
\] \footnotesize

\begin{Shaded}
\begin{Highlighting}[]
\FunctionTok{svycontrast}\NormalTok{(tab\_Sex\_Pobr, }
            \FunctionTok{quote}\NormalTok{((}\StringTok{\textasciigrave{}}\AttributeTok{interaction(Sex, pobreza)Male.1}\StringTok{\textasciigrave{}}\SpecialCharTok{/}
                     \StringTok{\textasciigrave{}}\AttributeTok{interaction(Sex, pobreza)Female.1}\StringTok{\textasciigrave{}}\NormalTok{)}\SpecialCharTok{/}
\NormalTok{                  (}\StringTok{\textasciigrave{}}\AttributeTok{interaction(Sex, pobreza)Male.0}\StringTok{\textasciigrave{}}\SpecialCharTok{/}
                   \StringTok{\textasciigrave{}}\AttributeTok{interaction(Sex, pobreza)Female.0}\StringTok{\textasciigrave{}}\NormalTok{)))}
\end{Highlighting}
\end{Shaded}

\begin{verbatim}
##          nlcon  SE
## contrast  1.02 0.1
\end{verbatim}
\end{frame}

\begin{frame}[fragile]{Contrastes}
\protect\hypertarget{contrastes}{}
El interés ahora es realizar en contraste de proporciones, por ejemplo:
\(\hat{p}_F - \hat{p}_M\)

\begin{Shaded}
\begin{Highlighting}[]
\NormalTok{(tab\_sex\_pobreza }\OtherTok{\textless{}{-}} \FunctionTok{svyby}\NormalTok{(}\SpecialCharTok{\textasciitilde{}}\NormalTok{pobreza, }\SpecialCharTok{\textasciitilde{}}\NormalTok{Sex, }
\NormalTok{                          diseno , }
\NormalTok{                      svymean, }\AttributeTok{na.rm=}\NormalTok{T,}
                      \AttributeTok{covmat =} \ConstantTok{TRUE}\NormalTok{,}
                      \AttributeTok{vartype =} \FunctionTok{c}\NormalTok{(}\StringTok{"se"}\NormalTok{, }\StringTok{"ci"}\NormalTok{)))}
\end{Highlighting}
\end{Shaded}

\begin{longtable}[]{@{}llrrrr@{}}
\toprule
& Sex & pobreza & se & ci\_l & ci\_u \\
\midrule
\endhead
Female & Female & 0.3892 & 0.0316 & 0.3273 & 0.4512 \\
Male & Male & 0.3946 & 0.0366 & 0.3228 & 0.4664 \\
\bottomrule
\end{longtable}

\begin{itemize}[<+->]
\tightlist
\item
  \emph{Paso 1:} Calcular la diferencia de estimaciones
\end{itemize}

\begin{Shaded}
\begin{Highlighting}[]
\FloatTok{0.3892} \SpecialCharTok{{-}} \FloatTok{0.3946}          
\end{Highlighting}
\end{Shaded}

\begin{verbatim}
## [1] -0.0054
\end{verbatim}
\end{frame}

\begin{frame}[fragile]{Contrastes de la diferencia de proporciones}
\protect\hypertarget{contrastes-de-la-diferencia-de-proporciones}{}
Con la función \texttt{vcov} obtiene la matriz de covarianzas

\begin{Shaded}
\begin{Highlighting}[]
\FunctionTok{library}\NormalTok{(kableExtra)}
\FunctionTok{vcov}\NormalTok{(tab\_sex\_pobreza)}\SpecialCharTok{\%\textgreater{}\%} \FunctionTok{data.frame}\NormalTok{() }\SpecialCharTok{\%\textgreater{}\%} 
  \FunctionTok{kable}\NormalTok{(}\AttributeTok{digits =} \DecValTok{10}\NormalTok{,}
        \AttributeTok{format.args =} \FunctionTok{list}\NormalTok{(}\AttributeTok{scientific =} \ConstantTok{FALSE}\NormalTok{))}
\end{Highlighting}
\end{Shaded}

\begin{tabular}{l|r|r}
\hline
  & Female & Male\\
\hline
Female & 0.0009983 & 0.0009183\\
\hline
Male & 0.0009183 & 0.0013416\\
\hline
\end{tabular}

\begin{itemize}[<+->]
\tightlist
\item
  \emph{Paso 2:} El cálculo del error estándar es:
\end{itemize}

\begin{Shaded}
\begin{Highlighting}[]
\FunctionTok{sqrt}\NormalTok{(}\FloatTok{0.0009983} \SpecialCharTok{+} \FloatTok{0.0013416} \SpecialCharTok{{-}} \DecValTok{2}\SpecialCharTok{*}\FloatTok{0.0009183}\NormalTok{)}
\end{Highlighting}
\end{Shaded}

\begin{verbatim}
## [1] 0.02243
\end{verbatim}
\end{frame}

\begin{frame}[fragile]{Contrastes de la diferencia de proporciones en R}
\protect\hypertarget{contrastes-de-la-diferencia-de-proporciones-en-r}{}
Para realizar la diferencia de proporciones se hace uso de la función
\texttt{svycontrast}.

\begin{Shaded}
\begin{Highlighting}[]
\FunctionTok{svycontrast}\NormalTok{(tab\_sex\_pobreza,}
            \FunctionTok{list}\NormalTok{(}\AttributeTok{diff\_Sex =} \FunctionTok{c}\NormalTok{(}\DecValTok{1}\NormalTok{, }\SpecialCharTok{{-}}\DecValTok{1}\NormalTok{))) }\SpecialCharTok{\%\textgreater{}\%}
  \FunctionTok{data.frame}\NormalTok{()}
\end{Highlighting}
\end{Shaded}

\begin{tabular}{l|r|r}
\hline
  & contrast & diff\_Sex\\
\hline
diff\_Sex & -0.0053 & 0.0224\\
\hline
\end{tabular}
\end{frame}

\begin{frame}[fragile]{Contrastes de la diferencia de proporciones}
\protect\hypertarget{contrastes-de-la-diferencia-de-proporciones-1}{}
Diferencia en desempleo por sexo.

\begin{Shaded}
\begin{Highlighting}[]
\NormalTok{(tab\_sex\_desempleo }\OtherTok{\textless{}{-}} \FunctionTok{svyby}\NormalTok{(}
  \SpecialCharTok{\textasciitilde{}}\NormalTok{desempleo, }\SpecialCharTok{\textasciitilde{}}\NormalTok{Sex, }
\NormalTok{    diseno }\SpecialCharTok{\%\textgreater{}\%} \FunctionTok{filter}\NormalTok{(}\SpecialCharTok{!}\FunctionTok{is.na}\NormalTok{(desempleo)) , }
\NormalTok{     svymean, }\AttributeTok{na.rm=}\NormalTok{T, }\AttributeTok{covmat =} \ConstantTok{TRUE}\NormalTok{,}
     \AttributeTok{vartype =} \FunctionTok{c}\NormalTok{(}\StringTok{"se"}\NormalTok{, }\StringTok{"ci"}\NormalTok{)))}
\end{Highlighting}
\end{Shaded}

\begin{tabular}{l|l|r|r|r|r}
\hline
  & Sex & desempleo & se & ci\_l & ci\_u\\
\hline
Female & Female & 0.0217 & 0.0056 & 0.0107 & 0.0326\\
\hline
Male & Male & 0.0678 & 0.0122 & 0.0440 & 0.0917\\
\hline
\end{tabular}

\begin{itemize}[<+->]
\tightlist
\item
  \emph{Paso 1}: Diferencia de las estimaciones
\end{itemize}

\begin{Shaded}
\begin{Highlighting}[]
\FloatTok{0.02169} \SpecialCharTok{{-}} \FloatTok{0.06783}   
\end{Highlighting}
\end{Shaded}

\begin{verbatim}
## [1] -0.04614
\end{verbatim}
\end{frame}

\begin{frame}[fragile]{Contrastes de la diferencia de proporciones}
\protect\hypertarget{contrastes-de-la-diferencia-de-proporciones-2}{}
Estimación de la matriz de covarianza:

\begin{Shaded}
\begin{Highlighting}[]
\FunctionTok{vcov}\NormalTok{(tab\_sex\_desempleo) }\SpecialCharTok{\%\textgreater{}\%} \FunctionTok{data.frame}\NormalTok{() }\SpecialCharTok{\%\textgreater{}\%} 
  \FunctionTok{kable}\NormalTok{(}\AttributeTok{digits =} \DecValTok{10}\NormalTok{,}
        \AttributeTok{format.args =} \FunctionTok{list}\NormalTok{(}\AttributeTok{scientific =} \ConstantTok{FALSE}\NormalTok{))}
\end{Highlighting}
\end{Shaded}

\begin{tabular}{l|r|r}
\hline
  & Female & Male\\
\hline
Female & 0.00003114 & 0.00002081\\
\hline
Male & 0.00002081 & 0.00014789\\
\hline
\end{tabular}

\begin{itemize}[<+->]
\tightlist
\item
  \emph{Paso 2}: Estimación del error estándar.
\end{itemize}

\begin{Shaded}
\begin{Highlighting}[]
\FunctionTok{sqrt}\NormalTok{(}\FloatTok{0.00003114}  \SpecialCharTok{+} \FloatTok{0.00014789} \SpecialCharTok{{-}} \DecValTok{2}\SpecialCharTok{*}\FloatTok{0.00002081}\NormalTok{)}
\end{Highlighting}
\end{Shaded}

\begin{verbatim}
## [1] 0.01172
\end{verbatim}
\end{frame}

\begin{frame}[fragile]{Contrastes de la diferencia de proporciones en R}
\protect\hypertarget{contrastes-de-la-diferencia-de-proporciones-en-r-1}{}
Siguiendo el ejemplo anterior se tiene que:

\begin{Shaded}
\begin{Highlighting}[]
\FunctionTok{svycontrast}\NormalTok{(tab\_sex\_desempleo,}
            \FunctionTok{list}\NormalTok{(}\AttributeTok{diff\_Sex =} \FunctionTok{c}\NormalTok{(}\SpecialCharTok{{-}}\DecValTok{1}\NormalTok{, }\DecValTok{1}\NormalTok{))) }\SpecialCharTok{\%\textgreater{}\%}
  \FunctionTok{data.frame}\NormalTok{()}
\end{Highlighting}
\end{Shaded}

\begin{tabular}{l|r|r}
\hline
  & contrast & diff\_Sex\\
\hline
diff\_Sex & 0.0461 & 0.0117\\
\hline
\end{tabular}
\end{frame}

\begin{frame}[fragile]{Contrastes}
\protect\hypertarget{contrastes-1}{}
La tabla de desempleo por región se obtiene como:

\begin{Shaded}
\begin{Highlighting}[]
\NormalTok{(tab\_region\_desempleo }\OtherTok{\textless{}{-}} \FunctionTok{svyby}\NormalTok{(}
  \SpecialCharTok{\textasciitilde{}}\NormalTok{desempleo, }\SpecialCharTok{\textasciitilde{}}\NormalTok{Region, }
\NormalTok{    diseno }\SpecialCharTok{\%\textgreater{}\%} \FunctionTok{filter}\NormalTok{(}\SpecialCharTok{!}\FunctionTok{is.na}\NormalTok{(desempleo)) , }
\NormalTok{    svymean, }\AttributeTok{na.rm=}\NormalTok{T, }\AttributeTok{covmat =} \ConstantTok{TRUE}\NormalTok{,}
     \AttributeTok{vartype =} \FunctionTok{c}\NormalTok{(}\StringTok{"se"}\NormalTok{, }\StringTok{"ci"}\NormalTok{)))}
\end{Highlighting}
\end{Shaded}

\begin{tabular}{l|l|r|r|r|r}
\hline
  & Region & desempleo & se & ci\_l & ci\_u\\
\hline
Norte & Norte & 0.0488 & 0.0200 & 0.0095 & 0.0880\\
\hline
Sur & Sur & 0.0656 & 0.0238 & 0.0191 & 0.1122\\
\hline
Centro & Centro & 0.0387 & 0.0124 & 0.0144 & 0.0630\\
\hline
Occidente & Occidente & 0.0400 & 0.0123 & 0.0159 & 0.0641\\
\hline
Oriente & Oriente & 0.0295 & 0.0126 & 0.0049 & 0.0541\\
\hline
\end{tabular}
\end{frame}

\begin{frame}{Creado una matriz de contrastes}
\protect\hypertarget{creado-una-matriz-de-contrastes}{}
Ahora, el interés es realizar los contrastes siguientes para desempleo:

\begin{itemize}[<+->]
\tightlist
\item
  \$\hat{p}\emph{\{Norte\} - \hat{p}}\{Centro\} = 0.01004 \$,
\item
  \$\hat{p}\emph{\{Sur\}-\hat{p}}\{Centro\} = 0.02691 \$\\
\item
  \$\hat{p}\emph{\{Occidente\}-\hat{p}}\{Oriente\} = 0.01046 \$
\end{itemize}

Escrita de forma matricial es:

\[
\left[\begin{array}{ccccc}
1 & 0 & -1 & 0 & 0\\
0 & 1 & -1 & 0 & 0\\
0 & 0 & 0 & 1 & -1
\end{array}\right]
\]
\end{frame}

\begin{frame}[fragile]{Contrastes múltiples}
\protect\hypertarget{contrastes-muxfaltiples}{}
\begin{Shaded}
\begin{Highlighting}[]
\FunctionTok{vcov}\NormalTok{(tab\_region\_desempleo)}\SpecialCharTok{\%\textgreater{}\%}
  \FunctionTok{data.frame}\NormalTok{() }\SpecialCharTok{\%\textgreater{}\%} 
  \FunctionTok{kable}\NormalTok{(}\AttributeTok{digits =} \DecValTok{10}\NormalTok{,}
        \AttributeTok{format.args =} \FunctionTok{list}\NormalTok{(}\AttributeTok{scientific =} \ConstantTok{FALSE}\NormalTok{))}
\end{Highlighting}
\end{Shaded}
\end{frame}

\begin{frame}[fragile]{Contrastes múltiples}
\protect\hypertarget{contrastes-muxfaltiples-1}{}
\scriptsize

\begin{tabular}{l|r|r|r|r|r}
\hline
  & Norte & Sur & Centro & Occidente & Oriente\\
\hline
Norte & 0.0004009 & 0.0000000 & 0.0000000 & 0.0000000 & 0.000000\\
\hline
Sur & 0.0000000 & 0.0005641 & 0.0000000 & 0.0000000 & 0.000000\\
\hline
Centro & 0.0000000 & 0.0000000 & 0.0001538 & 0.0000000 & 0.000000\\
\hline
Occidente & 0.0000000 & 0.0000000 & 0.0000000 & 0.0001512 & 0.000000\\
\hline
Oriente & 0.0000000 & 0.0000000 & 0.0000000 & 0.0000000 & 0.000158\\
\hline
\end{tabular}

\begin{Shaded}
\begin{Highlighting}[]
\FunctionTok{sqrt}\NormalTok{(}\FloatTok{0.0002981} \SpecialCharTok{+} \FloatTok{0.0002884} \SpecialCharTok{{-}} \DecValTok{2}\SpecialCharTok{*}\DecValTok{0}\NormalTok{) ;}
\end{Highlighting}
\end{Shaded}

\begin{verbatim}
## [1] 0.02422
\end{verbatim}

\begin{Shaded}
\begin{Highlighting}[]
\FunctionTok{sqrt}\NormalTok{(}\FloatTok{0.0001968} \SpecialCharTok{+} \FloatTok{0.0002884} \SpecialCharTok{{-}} \DecValTok{2}\SpecialCharTok{*}\DecValTok{0}\NormalTok{);}
\end{Highlighting}
\end{Shaded}

\begin{verbatim}
## [1] 0.02203
\end{verbatim}

\begin{Shaded}
\begin{Highlighting}[]
\FunctionTok{sqrt}\NormalTok{(}\FloatTok{0.0001267} \SpecialCharTok{+} \FloatTok{0.0004093} \SpecialCharTok{{-}} \DecValTok{2}\SpecialCharTok{*}\DecValTok{0}\NormalTok{)}
\end{Highlighting}
\end{Shaded}

\begin{verbatim}
## [1] 0.02315
\end{verbatim}
\end{frame}

\begin{frame}[fragile]{Creado una matriz de contrastes en \texttt{R}}
\protect\hypertarget{creado-una-matriz-de-contrastes-en-r}{}
\scriptsize

\begin{Shaded}
\begin{Highlighting}[]
\FunctionTok{svycontrast}\NormalTok{(tab\_region\_desempleo, }\FunctionTok{list}\NormalTok{(}
           \AttributeTok{Norte\_sur =} \FunctionTok{c}\NormalTok{(}\DecValTok{1}\NormalTok{, }\DecValTok{0}\NormalTok{, }\SpecialCharTok{{-}}\DecValTok{1}\NormalTok{, }\DecValTok{0}\NormalTok{, }\DecValTok{0}\NormalTok{),}
          \AttributeTok{Sur\_centro =} \FunctionTok{c}\NormalTok{(}\DecValTok{0}\NormalTok{, }\DecValTok{1}\NormalTok{, }\SpecialCharTok{{-}}\DecValTok{1}\NormalTok{, }\DecValTok{0}\NormalTok{, }\DecValTok{0}\NormalTok{),}
   \AttributeTok{Occidente\_Oriente =} \FunctionTok{c}\NormalTok{(}\DecValTok{0}\NormalTok{, }\DecValTok{0}\NormalTok{, }\DecValTok{0}\NormalTok{, }\DecValTok{1}\NormalTok{, }\SpecialCharTok{{-}}\DecValTok{1}\NormalTok{)}
\NormalTok{            )) }\SpecialCharTok{\%\textgreater{}\%} \FunctionTok{data.frame}\NormalTok{()}
\end{Highlighting}
\end{Shaded}

\begin{tabular}{l|r|r}
\hline
  & contrast & SE\\
\hline
Norte\_sur & 0.0100 & 0.0236\\
\hline
Sur\_centro & 0.0269 & 0.0268\\
\hline
Occidente\_Oriente & 0.0105 & 0.0176\\
\hline
\end{tabular}
\end{frame}

\begin{frame}[fragile]{Ejercicio.}
\protect\hypertarget{ejercicio.}{}
Repetir el contraste anterior para pobreza.

\begin{Shaded}
\begin{Highlighting}[]
\NormalTok{(tab\_region\_pobreza }\OtherTok{\textless{}{-}} \FunctionTok{svyby}\NormalTok{(}
  \SpecialCharTok{\textasciitilde{}}\NormalTok{pobreza, }\SpecialCharTok{\textasciitilde{}}\NormalTok{Region, }
\NormalTok{    diseno }\SpecialCharTok{\%\textgreater{}\%} \FunctionTok{filter}\NormalTok{(}\SpecialCharTok{!}\FunctionTok{is.na}\NormalTok{(desempleo)) , }
\NormalTok{     svymean, }\AttributeTok{na.rm=}\NormalTok{T, }\AttributeTok{covmat =} \ConstantTok{TRUE}\NormalTok{,}
      \AttributeTok{vartype =} \FunctionTok{c}\NormalTok{(}\StringTok{"se"}\NormalTok{, }\StringTok{"ci"}\NormalTok{)))}
\end{Highlighting}
\end{Shaded}

\begin{tabular}{l|l|r|r|r|r}
\hline
  & Region & pobreza & se & ci\_l & ci\_u\\
\hline
Norte & Norte & 0.3263 & 0.0480 & 0.2322 & 0.4204\\
\hline
Sur & Sur & 0.2947 & 0.0479 & 0.2007 & 0.3886\\
\hline
Centro & Centro & 0.3234 & 0.0721 & 0.1820 & 0.4647\\
\hline
Occidente & Occidente & 0.3673 & 0.0440 & 0.2811 & 0.4536\\
\hline
Oriente & Oriente & 0.3871 & 0.0916 & 0.2075 & 0.5666\\
\hline
\end{tabular}
\end{frame}

\begin{frame}{Creado una matriz de contrastes}
\protect\hypertarget{creado-una-matriz-de-contrastes-1}{}
Ahora, el interés es realizar los contrastes siguientes para pobreza:

\begin{itemize}[<+->]
\tightlist
\item
  \$\hat{p}\emph{\{Norte\} - \hat{p}}\{Centro\} \$,
\item
  \$\hat{p}\emph{\{Sur\}-\hat{p}}\{Centro\} \$\\
\item
  \(\hat{p}_{Occidente}-\hat{p}_{Oriente}\)
\end{itemize}

Escrita de forma matricial es:

\[
\left[\begin{array}{ccccc}
1 & 0 & -1 & 0 & 0\\
0 & 1 & -1 & 0 & 0\\
0 & 0 & 0 & 1 & -1
\end{array}\right]
\]
\end{frame}

\begin{frame}[fragile]{Creado una matriz de contrastes en \texttt{R}}
\protect\hypertarget{creado-una-matriz-de-contrastes-en-r-1}{}
\scriptsize

\begin{Shaded}
\begin{Highlighting}[]
\FunctionTok{svycontrast}\NormalTok{(tab\_region\_pobreza, }\FunctionTok{list}\NormalTok{(}
           \AttributeTok{Norte\_sur =} \FunctionTok{c}\NormalTok{(}\DecValTok{1}\NormalTok{, }\DecValTok{0}\NormalTok{, }\SpecialCharTok{{-}}\DecValTok{1}\NormalTok{, }\DecValTok{0}\NormalTok{, }\DecValTok{0}\NormalTok{),}
          \AttributeTok{Sur\_centro =} \FunctionTok{c}\NormalTok{(}\DecValTok{0}\NormalTok{, }\DecValTok{1}\NormalTok{, }\SpecialCharTok{{-}}\DecValTok{1}\NormalTok{, }\DecValTok{0}\NormalTok{, }\DecValTok{0}\NormalTok{),}
   \AttributeTok{Occidente\_Oriente =} \FunctionTok{c}\NormalTok{(}\DecValTok{0}\NormalTok{, }\DecValTok{0}\NormalTok{, }\DecValTok{0}\NormalTok{, }\DecValTok{1}\NormalTok{, }\SpecialCharTok{{-}}\DecValTok{1}\NormalTok{)}
\NormalTok{            )) }\SpecialCharTok{\%\textgreater{}\%} \FunctionTok{data.frame}\NormalTok{()}
\end{Highlighting}
\end{Shaded}

\begin{tabular}{l|r|r}
\hline
  & contrast & SE\\
\hline
Norte\_sur & 0.0029 & 0.0866\\
\hline
Sur\_centro & -0.0287 & 0.0866\\
\hline
Occidente\_Oriente & -0.0197 & 0.1016\\
\hline
\end{tabular}
\end{frame}

\begin{frame}{¡Gracias!}
\protect\hypertarget{gracias}{}
\emph{Email}:
\href{mailto:andres.gutierrez@cepal.org}{\nolinkurl{andres.gutierrez@cepal.org}}
\end{frame}

\end{document}
