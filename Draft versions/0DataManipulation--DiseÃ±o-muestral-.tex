% Options for packages loaded elsewhere
\PassOptionsToPackage{unicode}{hyperref}
\PassOptionsToPackage{hyphens}{url}
%
\documentclass[
]{article}
\title{Análisis de encuestas de hogares con R}
\usepackage{etoolbox}
\makeatletter
\providecommand{\subtitle}[1]{% add subtitle to \maketitle
  \apptocmd{\@title}{\par {\large #1 \par}}{}{}
}
\makeatother
\subtitle{Módulo 0: Introducción a R y dplyr}
\author{}
\date{\vspace{-2.5em}CEPAL - Unidad de Estadísticas Sociales}

\usepackage{amsmath,amssymb}
\usepackage{lmodern}
\usepackage{iftex}
\ifPDFTeX
  \usepackage[T1]{fontenc}
  \usepackage[utf8]{inputenc}
  \usepackage{textcomp} % provide euro and other symbols
\else % if luatex or xetex
  \usepackage{unicode-math}
  \defaultfontfeatures{Scale=MatchLowercase}
  \defaultfontfeatures[\rmfamily]{Ligatures=TeX,Scale=1}
\fi
% Use upquote if available, for straight quotes in verbatim environments
\IfFileExists{upquote.sty}{\usepackage{upquote}}{}
\IfFileExists{microtype.sty}{% use microtype if available
  \usepackage[]{microtype}
  \UseMicrotypeSet[protrusion]{basicmath} % disable protrusion for tt fonts
}{}
\makeatletter
\@ifundefined{KOMAClassName}{% if non-KOMA class
  \IfFileExists{parskip.sty}{%
    \usepackage{parskip}
  }{% else
    \setlength{\parindent}{0pt}
    \setlength{\parskip}{6pt plus 2pt minus 1pt}}
}{% if KOMA class
  \KOMAoptions{parskip=half}}
\makeatother
\usepackage{xcolor}
\IfFileExists{xurl.sty}{\usepackage{xurl}}{} % add URL line breaks if available
\IfFileExists{bookmark.sty}{\usepackage{bookmark}}{\usepackage{hyperref}}
\hypersetup{
  pdftitle={Análisis de encuestas de hogares con R},
  hidelinks,
  pdfcreator={LaTeX via pandoc}}
\urlstyle{same} % disable monospaced font for URLs
\usepackage[margin=1in]{geometry}
\usepackage{graphicx}
\makeatletter
\def\maxwidth{\ifdim\Gin@nat@width>\linewidth\linewidth\else\Gin@nat@width\fi}
\def\maxheight{\ifdim\Gin@nat@height>\textheight\textheight\else\Gin@nat@height\fi}
\makeatother
% Scale images if necessary, so that they will not overflow the page
% margins by default, and it is still possible to overwrite the defaults
% using explicit options in \includegraphics[width, height, ...]{}
\setkeys{Gin}{width=\maxwidth,height=\maxheight,keepaspectratio}
% Set default figure placement to htbp
\makeatletter
\def\fps@figure{htbp}
\makeatother
\setlength{\emergencystretch}{3em} % prevent overfull lines
\providecommand{\tightlist}{%
  \setlength{\itemsep}{0pt}\setlength{\parskip}{0pt}}
\setcounter{secnumdepth}{-\maxdimen} % remove section numbering
\ifLuaTeX
  \usepackage{selnolig}  % disable illegal ligatures
\fi

\begin{document}
\maketitle

\hypertarget{introducciuxf3n}{%
\section{Introducción}\label{introducciuxf3n}}

Las encuestas de hogares son uno de los instrumentos más importantes
para hacer seguimiento a los indicadores de los Objetivos de Desarrollo
Sostenible (ODS, por sus siglas) en el marco de la agenda 2030. Dada la
importancia que tiene estas encuestas en la política pública de cada
país, es necesario que los resultados que se obtengan de ellas sean lo
más precisos y confiables posibles. En este sentido, las herramientas
estadísticas utilizadas para obtener dichos resultados deben ser lo más
robustas posibles. Particularmente, el diseño de muestreo utilizando,
sin lugar a dudas, es un diseño de muestreo complejo. Entiéndase esto
como como aquello diseños de muestreo en los cuales las unidades
experimentales no pueden ser seleccionadas directamente del marco. Es
decir, aquellos diseños que contienen más de una etapa, estratificación,
conglomerados, etc.

El objetivo principal de este libro es presentar los conceptos
necesarios para hacer un análisis de encuestas complejas enfocadas en
las dinámicas de los hogares. Particularmente, se presenta una guía
práctica para analizar encuestas complejas usando R. Es por esto que, la
dinámica que se trabaja en este texto es guiar al lector a cómo realizar
un análisis completo de una encuesta compleja usando el software
estadístico R con el paquete survey. En ese sentido, todos los ejemplos,
tablas y gráficos que se presentan en este libro se producen con R, y
los códigos computacionales para reproducir estarán disponibles para
replicarlos. Se decide utilizar el software estadístico R para hacer los
análisis puesto que, es un software de código abierto, lo que permite
que cualquier investigador o instituto estadístico tenga acceso a él y
es muy conocido y utilizado por el gremio estadístico, lo que lo hace
conveniente para la enseñanza.

El lector encontrará en este texto la siguiente estructura. En el
capítulo 2 se describen los conceptos básicos de una encuesta compleja
fundamentales para la correcta definición del diseño muestral en el
entorno de las encuestas de hogares. En el capítulo 3 y 4 se definen los
conceptos de variables aleatoria continua y discretas respectivamente en
el contexto del muestreo probabilístico y, en el capítulo 5 se muestra
como ajustar modelos de regresión lineal utilizando varibales discretas
y continuas empleando las herramientas del muestreo probabilístico. En
el capítulo 6 se presentan las herramientas para ajustar modelos de
regresión logística los cuales son fundamentales en el análisis de
encuestas de hogares.

Ahora bien, en los análisis estadísticos no solo son requeridos los
modelos de regresión lineales, también, por la misma naturaleza de las
variables capturadas en una encuesta de hogares, es necesario el ajuste
de modelos lineales generalizados y multiniveles, estos conceptos son
trabajados en el capítulo 7 y 8 respectivamente.

Ahora bien, dada la pandemía la no respuesta en encuestas de hogares a
aumentado de manera importante en los últimos años por lo que, es
necesario recurrir a técnicas de imputación para la información no
capturada en el trabajo de campo. Esta temática es trabajada en el
capítulo 9. Por último, la presentación gráfica de los resultados en una
encuesta de hogares derá abordada en el capítulo 10.

\hypertarget{conceptos-buxe1sicos-en-encuestas-de-hogares}{%
\section{Conceptos básicos en encuestas de
hogares}\label{conceptos-buxe1sicos-en-encuestas-de-hogares}}

En este capítulo se presentan los conceptos básicos A continuación, se
presentan

\hypertarget{universo-de-estudio}{%
\subsection{Universo de estudio}\label{universo-de-estudio}}

\begin{itemize}
\tightlist
\item
  El término encuesta se encuentra directamente relacionado con una
  población finita compuesta de individuos a los cuales es necesario
  entrevistar.
\item
  Este conjunto de unidades de interés recibe el nombre de
  \emph{población objetivo} o \emph{universo} y sobre ellas se obtiene
  la información de interés para el estudio.
\item
  Por ejemplo, \emph{la Encuesta Nacional de Empleo y Desempleo} de
  Ecuador define su población objetivo como todas las personas mayores
  de 10 años residentes en viviendas particulares en Ecuador.
\end{itemize}

\hypertarget{unidades-de-anuxe1lisis}{%
\subsection{Unidades de análisis}\label{unidades-de-anuxe1lisis}}

\begin{itemize}
\tightlist
\item
  Corresponden a los diferentes niveles de desagregación establecidos
  para consolidar el diseño probabilístico y sobre los que se presentan
  los resultados de interés.
\item
  En México, la \emph{Encuesta Nacional de Ingresos y Gastos de los
  Hogares} define como unidades de análisis el ámbito al que pertenece
  la vivienda, urbano alto, complemento urbano y rural.
\item
  La \emph{Gran Encuesta Integrada de Hogres} de Colombia tiene
  cobertura nacional y sus unidades de análisis están definidas por 13
  grandes ciudades junto con sus áreas metropolitanas.
\end{itemize}

\hypertarget{unidades-de-muestreo}{%
\subsection{Unidades de muestreo}\label{unidades-de-muestreo}}

\begin{itemize}
\item
  El diseño de una encuesta de hogares en América Latina plantea la
  necesidad de seleccionar en varias etapas ciertas \emph{unidades de
  muestreo} que sirven como medio para seleccionar finalmente a los
  hogares que participarán de la muestra.
\item
  La \emph{Pesquisa Nacional por Amostra de Domicilios} en Brasil se
  realiza por medio de una muestra de viviendas en tres etapas.
\end{itemize}

\hypertarget{unidades-de-muestreo-en-pnad}{%
\subsection{Unidades de muestreo en
PNAD}\label{unidades-de-muestreo-en-pnad}}

\begin{enumerate}
\def\labelenumi{\arabic{enumi}.}
\tightlist
\item
  Las unidades primarias de muestreo (UPM) son los municipios,
\item
  Las unidades secundarias de muestreo (USM) son los sectores censales,
  que conforman una malla territorial conformada en el último Censo
  Demográfico.
\item
  Las últimas unidades en ser seleccionadas son las viviendas.
\end{enumerate}

\hypertarget{marcos-de-muestreo}{%
\subsection{Marcos de muestreo}\label{marcos-de-muestreo}}

\begin{itemize}
\tightlist
\item
  Para realizar el proceso de selección sistemática de los hogares es
  necesario contar con un marco de muestreo que sirva de \emph{link}
  entre los hogares y las unidades de muestreo y que permita tener
  acceso a la población de interés.
\item
  El marco de muestreo debe permitir identificar y ubicar a todos los
  hogares que conforman la población objetivo.
\item
  Los marcos de muestreo más utilizados en este tipo de encuestas son de
  áreas geográficas que vinculan directamente a los hogares o personas.
\end{itemize}

\hypertarget{ejemplo-de-costa-rica}{%
\subsection{Ejemplo de Costa Rica}\label{ejemplo-de-costa-rica}}

\begin{itemize}
\tightlist
\item
  La \emph{Encuesta Nacional de Hogares} de utiliza un marco muestral
  construido a partir de los censos nacionales de población y vivienda
  de 2011.
\item
  Corresponde a un marco de áreas en donde sus unidades son superficies
  geográficas asociadas con las viviendas.
\item
  Permite la definición de UPM con 150 viviendas en las zonas urbanas y
  100 viviendas en las zonas rurales.
\item
  El marco está conformado por 10461 UPM (64.5\% urbanas y 35.5\%
  rurales).
\end{itemize}

\hypertarget{objetivos-de-la-pnad}{%
\subsection{Objetivos de la PNAD}\label{objetivos-de-la-pnad}}

\begin{itemize}
\tightlist
\item
  La \emph{Pesquisa Nacional por Amostra de Domicílios Contínua} es
  implementada cada trimestre por el \emph{Instituto Brasileiro de
  Geografia e Estatística}.
\item
  Su objetivo es producir información básica para el estudio de la
  evolución económica de Brasil y la publicación continua de indicadores
  demográficos.
\item
  Los constructos de ingreso, gastos y empleo son evaluados de forma
  continua.
\item
  Además evalúa temas de vivienda, migración de los individuos del
  hogar, trabajo infantil, fecundidad, salud y seguridad alimentaria,
  uso de las tecnologías de información, transferencias de renta, uso
  del tiempo, entre otros.
\end{itemize}

\hypertarget{gracias}{%
\subsection{¡Gracias!}\label{gracias}}

\emph{Email}:
\href{mailto:andres.gutierrez@cepal.org}{\nolinkurl{andres.gutierrez@cepal.org}}

\end{document}
